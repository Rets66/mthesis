\section{評価}
本章では,提案システムのDNS Tunnelingに対する機能性について評価した結果を示す.
また,ネットワークに対する影響を測るために,3つの特性(RTT・パケットサイズ・トラフィック量)について既存システムと比較した結果を述べる.
評価にあたり,提案システムを実装した.
はじめに,システム実装で採用したアルゴリズムに
本章では,提案システムのDNS Tunnelingに対する機能性評価とネットワークへの影響度について各要素ごとに関する特性評価を示す.
\subsection{実装}
本節では,提案システムの機能性および特性評価を目的として実装した
\subsection{実験環境}
\subsection{DNS Tunneling}
\subsubsection{DNS Exfiltration}
本項では,提案システムのDNS Exfiltrationへの機能性について評価した結果を述べる.
評価には,実装したシステムを
\subsubsection{DNS Infiltration}
\subsection{要素ごとの特性}
本節では,SORESの特性について既存のDNSと比較に基づき評価する.

SORESでは,スタブリゾルバからの名前解決クエリのうち,そのヘッダー情報に基づき識別子を算出することで,この識別子に関連づけられたレコード情報を操作する仕組みに基づき名前解決機能が動作する.
DNSでは,ゾーンをドメインごとに分割し,フルサービスリゾルバはルート権威サーバから目的の権威サーバに向かって再帰的に問い合わせ,権威サーバのアドレスを解決しながら,最終的にコンテンツを保持する権威サーバからコンテンツを取得する.
既存のDNSとSORESの特性を表~\ref{tab:diff_sores_dns}に示す.

\begin{table}[htb]
 \centering
  \begin{tabular}{ccc}
    \toprule
     & \textbf{DNS} & \textbf{SORES} \\
    \midrule
    \textbf{RTT} & 全ての権威サーバとのRTTの総和 & マネージャとのRTT  \\ \hline
		\textbf{ドメイン長(パケットサイズ)} & 変長(最大253byte) & 固定長(56byte)  \\ \hline
    \textbf{トランザクション回数} & ラベル数依存 & 1 \\ \hline
    \textbf{トラフィック量} & 線形増加 & 線形増加 \\ \hline
    \textbf{リクエスト速度} & N/A & ハッシュ計算分遅延 \\
    \bottomrule
  \end{tabular}
 \caption{DNSとSORESの特性比較}
 \label{tab:diff_sores_dns}
\end{table}


から識別子に基づき,コンテンツを保持するサーバに名前解決リクエストを投げる.
SORESは,フルサービスリゾルバにてスタブリゾルバから問い合わせられた名前解決命令に基づきレコード情報の識別子を算出する.
この識別子に基づき,フルサービスリゾルバからは,1ホップでコンテンツを保持するサーバに問い合わせる仕組みで動作する.
\subsubsection{RTT(Round Trip Time)}
本項では,SORESが識別子から一意にレコード情報にアクセス可能である点について,DNSとの比較評価を行う.
SORESの名前解決では,
従来のDNSでは,ラベルごとにゾーンが移譲されている場合,レコード情報を保持するノードまでのホップ数はラベル数nに比例する.
それに対して,SORESでは,識別子から一意にレコード情報を保持するマネージャノードを特定できるので,常にホップ数は2である.
このため,既存の名前解決システムより速度の向上が期待される.
% 実装の必要はない
% 既存のDNSにおけるラウンドトリップのうち,再帰問い合わせの最後の権威サーバRTTがSORESのフルリゾルバとマネージャのそれになる.
% 遠いものと近いもののRTTを用意する必要がある
\subsubsection{パケットサイズ}
% 実装する必要はない.理論で評価できる
% あえてするとすれば,フルサービスリゾルバとマネージャ間通信
% DDoSへの影響については,リクエストとレスポンスパケットのサイズからアンプ率に着目する
% アンプ率は,既存のDNSと基本的に変わらない.DNSSECがあるかないかについて議論するくらい
SORESでは,224bitを固定長とするコンテンツIDをシンボルとすることによって,レコード情報にアクセスする.
この仕組みの影響で,SORESのパケットは従来のパケットと比較して肥大する特性がある.
このため,送信元を目的ホストと偽装することで目的ホストの計算リソースを圧迫するDDoS攻撃に対して,脅威を高める可能性が予想される.
\subsubsection{トラフィック量}
% SORESについては,実装の必要性はない
% ただし,DNSについてトラフィックデータを収集する必要がある
% 必要なデータ
% * 一定期間のSORESトラフィック
% * 同一期間のDNSトラフィック
% 比較項目
% * 平均したドメイン長と
本項では,提案手法におけるトラフィックについて評価する.
評価において,
SORESでは,シンボル志向の名前解決メソッドによって,既存の再帰問い合わせによるメソッドよりも少ないトラフィックに抑えることが期待される.
SORESでは,クエリ数とトラフィック量は比例関係にある.
他方で,新たにマネージャ間通信という従来にはないトラフィックが発生する.
本項では,これトラフィックがネットワーク全体にどの程度影響を及ぼすのかについて評価する.
% マネージャが攻撃された場合の影響は,世界規模に影響する.
% 従来のドメインごとにゾーンが分離している設計と違い,一つのゾーンには様々な組織のドメインが管理されている.
% このゾーンを全て書き換えることができることは脅威である.

