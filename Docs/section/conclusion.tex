\section{結論}
% 機能を分離させることによって,クエリを用いてデータを転送する仕組みを無効化することを実現
% 他の名前解決システムと違い,クライアントサイドへの変更点が少ないためマイグレーションが期待される
本論文では,DNSトンネリングを抑止する名前解決システムを提案した.
DNSトンネリングが動作する仕組みを分析し,修正する対象として,スタブリゾルバから権威サーバまでクエリパケットが転送される仕組みに着目した.
提案システムDNS-TDは,権威サーバのコンテンツを保持およびクライアントからの問い合わせに応答する機能とコンテンツを編集する機能を2つサービスノードに分割させる設計をとった.
これによって,スタブリゾルバと任意の権威サーバ間でデータがやりとりされることが抑制され,DNSトンネリング通信発生の抑止に寄与する.
%また,DNSは現在のインターネットを支える基盤システムとして根付いており,大幅な修正が加えられた場合,接続障害が発生する可能性がある.
%提案システムは,フルサービスリゾルバと権威サーバにおける名前解決の仕組みのみの変更に抑えているため.現実的なマイグレーションが期待される.
評価では,実装した提案システムのプロトタイプ上での擬似トンネリング通信を発生させるシミュレーションテストに基づいて,トンネリング通信の抑止に対して本提案システムが有用であることを示した.
また,特性評価の結果から,提案システムの名前解決におけるクエリ転送回数がスタブリゾルバからのサーバまで2回である点は,他のどの名前解決システムよりも少ない.
すなわち,提案システムが高速な名前解決とトラフィック量を削減できるという特性を持っていることを明らかにした.
%また,提案システムの特性の評価を行い,提案システムが既存システムよりも少ないトラフィックでかつ高速な名前解決を実現できるという優位性があることを示した.

提案システムは,ドメインに依らないフラットな名前空間の範囲に基づきゾーンを分割する設計上,1つのサーバが保持するゾーンには複数のドメインが含まれる.
そのため,既存システムにおける権威サーバの障害では,その権威サーバが管理するドメインのみに影響が及ぶのに対し,提案システムではインターネット全体に影響が及ぶ潜在的な脅威がある.
今後の課題は,この脅威の対処として負荷分散の仕組みを検討することである.
