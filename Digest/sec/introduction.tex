\section{背景}
増加し続けるサイバー攻撃に対して,現在多くの組織は,SIEMのようなネットワークトラフィックを監視するシステムから発せられるアラートを処理することでセキュリティの脅威に対処している.
一方,機密情報の奪取や諜報活動を行う攻撃者は,そのような監視システムを迂回するために秘匿通信手法を用いることが知られている~\cite{mitre-custom-c2}.
このような攻撃に対して,検知の閾値や悪性モデルを調整するアプローチが考えられるが,誤検知とのトレードオフの関係にある.

秘匿通信の代表的な手法にDNSトンネリングがある.
DNS(Domain Name System)は,IPアドレスをはじめとしたドメイン名に関連づけられたリソースレコードを解決するシステムである.
この名前解決の機能によって,ユーザは識別しづらいIPアドレスを直接利用することなくサーバのリソースにアクセスすることができる.
インターネットの利活用において,名前解決の通信はメールの送受信やウェブ検索などの通信に先立って行われる.
すなわち,DNSのトラフィックをフィルタリングすることはインターネットの利活用に大きな影響を及ぼすため,容易にフィルタリングを行うことが困難であるという特性がある.
DNSトンネリングは,フィルタリングされにくいというDNSのこの特性を利用する.
DNSトンネリングでは,クライアントからサーバ方向のデータ転送にQnameを用い,その逆方向のデータ転送にはリソースレコードを使用する.
このように双方向にデータを転送できるDNSトンネリングは,ターゲット組織から取得したデータを外部に流出させる際の手段としてだけでなく,ターゲットネットワークに潜伏しているマルウェアに対するC2通信の手段として,サイバー攻撃で広く利用されている.

しかしDNSプロトコルは,その名前解決の仕組みは変更されずに現在まで使用され続けている.
このDNSトンネリングに対して,同一ドメインへの時間あたりのトラフィック頻度や問い合わせられるドメイン名のサブドメインにおける文字列の分布や長さといった特徴に基づく検知アプローチはこれまでに多数提案されている~\cite{born, cheng, liu, asaf, steadman, jawad}.
検知に基づくアプローチを取る先行研究では,DNSトンネリングの擬似通信として,Githubなどから入手可能なトンネリング実装プログラムが用いた評価が行われる.
しかし,擬似通信の発生に広く使われるIodine~\cite{iodine}やDNSCat2~\cite{dnscat2}といった実装は,インタラクティブシェル機能を目的としているため,時間あたりのトラフィック量が多く,パケットサイズも大きいという性質がある.
このような顕著な性質が現れるトンネリング実装に対して,パケットごとのインターバルを1ヶ月間などにトラフィック頻度を調整したり,正規のFQDNの平均の長さまで注入するデータ量を下げるなどのバイパス手法がある~\cite{asaf}.
これらバイパス手法を利用した場合,トンネリング実装の特徴に基づいた既存の検知手法を用いて検知することは困難である.

一方,これまでに多数の次世代名前解決システムは提案されてきているが,DNSトンネリング抑止を目的としたシステムは筆者が知りうる限り提案されていない.
そこで本研究では,秘匿通信手法であるDNSトンネリングの発生を抑止する次世代の名前解決システムを提案する.


\subsection{目的}
本研究の目的は,従来の名前解決システムを保ちながら,秘匿通信手法のDNSトンネリングを抑止する名前解決システムを開発することである.
既存のDNSによる名前解決システムは,現在のインターネットの根幹技術であるため,移行を念頭に設計されている必要がある.
%その名前解決エコシステムに大幅な変更を加えることは,高い導入コストが要求されるため望まれない.
例えば,期待されないシステムとしては,以下のようなものが予想される.
\begin{itemize}
 \setlength{\itemsep}{-0.5mm}
 \item DNSトンネリング抑止は実現されるが,既存の名前解決システムが実現されない
 \item DNSトンネリング抑止は実現されるが,既存システムからの大幅な変更が必要になり,未対応のコンピュータのインターネット接続に支障をきたす
\end{itemize}
以上のことを踏まえた既存システムとの互換性を確保しながら,目的の実現を目指す.

\subsection{貢献}
提案システムは,既存システムのクライアントサーバアーキテクチャに基づき,既存の再帰問い合わせの仕組みのみに改変処理を施すことで,DNSトンネリングを抑止する名前解決システムの実現させる.
提案システムは,既存クライアントに変更を加えずに秘匿通信としての機能を抑止することができるため,セキュアな名前解決システムとして広く一般に利用されることが期待される.

