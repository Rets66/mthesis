\section{考察}
第~\ref{sec:evaluation}章で示す通り,提案システムにおける権威サーバの機能の分離と識別子に基づくマネージャの探索手法によって,トンネリングを抑止しながら既存の名前解決システムの基盤を踏襲できることを明らかにした.
また,比較に基づく特性評価から,提案システムはトラフィック量の削減と名前解決の高速化が期待されることを示した.
本章では,提案システムにおける課題と有用性について考察する.

\subsection{提案システムをバイパスするトンネリング手法}
提案システムでは,権威サーバの機能を分離させ,コンテンツを提供するサーバをユーザのアクセスできる範囲から離すことによって,トンネリングの発生を抑止させている.
しかし,スタブリゾルバからのクエリがマネージャに転送されているため,細工したDNSクエリなどなんらかの手法によってマネージャのクエリログにアクセスされた場合にトンネリングとして悪用される潜在的な脅威はある.
このような脅威シナリオに対して,DNSクエリのパーサーアルゴリズムと以降で述べるDoS攻撃に対処することが重要である.
現在のところ,ユーザがマネージャにアクセスできない限り,既存システムにおけるトンネリング手法のようにデータ転送する手法は,クライアントからの問い合わせをマネージャが処理する機構に基づいて困難である.

%\subsection{マイグレーション}

\subsection{ハッシュ関数の寿命とシステムの継続性}
提案システムでは,コンテンツIDおよびドメインIDにはKeccakアルゴリズムに基づいたsha3ハッシュ関数によって算出されるメッセージダイジェストを用いている.
ハッシュ関数は,一方向性の性能は計算時間の長さに基づいて担保されている.
当然,計算リソースの向上に伴い提案システムのハッシュアルゴリズムが解析される可能性は十分にある.
副次的に,フルサービスリゾルバとサーバとのネットワークにおけるクエリパケットの秘匿化に寄与する効果があるが,提案システムにおける利用目的は他にある.
提案システムにおけるハッシュ関数の利用は,元のメッセージから相関のないダイジェストが算出される性質を主として利用している.
このため,元のメッセージを衝突する解析が可能になることと提案システムにおける名前解決の機能およびトンネリング抑止の機能に影響が及ぶことはない.

\subsection{課題}
他方で,提案システムでは,マネージャというサービスノードへの信頼に基づいて,ドメイン名とレコードタイプのペアに識別子を付与し,複数のマネージャが識別子が帰属する名前空間の範囲を協調的に管理することでコンテンツが管理される.
この設計では,マネージャに障害が発生した際には極めて広範囲に影響が及ぶことが懸念される.
特に,ユーザとの接続点が名前解決クエリにあるので,DoS攻撃\footnote{DoS攻撃: サービス不全攻撃.大量のクエリによってサーバの計算リソースを消費させることによってサービス提供を不全にする.}への耐性を備えることが極めて重要となると予想される.
このDoS攻撃の脅威に対することは,今後の課題である.

% また,なんらかの手法によって攻撃者がマネージャが保有するコンテンツにアクセスできるようになった場合には,コンテンツの書き換えや削除などの悪意行為の脅威があり,インターネット全体に影響が波及することが予想される.

%\subsection{マネージャドメインの選抜とプロバイダの配属}
% マネージャノードが悪意のあるユーザによって攻撃された際については,議論する必要があるのか
% - プロバイダからの問い合わせをパースする際にセキュリティ機構を備えることによって,対策を講じる必要性がある
%\subsection{マネージャノードの最適な数}
% 提案システムでは,全てのネットワークが統一的な名前解決基盤を使用することになるため,国家レベルで検閲が実施されている環境には極めて大きな影響を及ぼす
