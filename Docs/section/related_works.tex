\section{関連研究}
\label{sec:related-works}
本章では,次世代DNSとして提案されている名前解決システムのうち提案手法に関連するP2Pアーキテクチャによるもの紹介し,提案手法との差分およびDNSトンネリングに対する課題を示す.
%最後に,既存の検知手法および次世代名前解決システムの課題から,既存のシステムに迎合しながらDNSトンネリングを緩和する名前解決システムの必要性を明らかにする.
%また,新しいアーキテクチャを導入するとき 既存のシステムとのマイグレーションを考慮する必要がある点について,本提案手法がマイグレーションを考慮している設計である点について
\subsection{P2Pネットワークを利用した名前解決システム}
これまでに,DNSにおける〜の課題に対して,P2Pに基づいた名前解決システムは数多く提案されてきた.
\subsubsection{GNS}
\subsubsection{SocialNS}
\subsection{Blockchainを利用した名前解決システム}
\subsubsection{Namecoin}
%\subsubsection{フラットな名前空間に基づく名前解決システム}
\subsection{課題}
\label{sec:issue-past-works}
本節では,先行研究および新しいアーキテクチャに基づく名前解決システムにおけるDNSトンネリングへの課題を示す.
(大筋)検知に基づく手法は,誤検知が避けられない.ペイロードアナリシスに対しては,一回あたりの転送量を調整することでバイパスすることができる.
分析の対象がトラフィックの場合は,頻度を長期間に延長することで,ログファイルの肥大によるストレージの過去のログファイルとの分析コストを重くなり分析の隙間をバイパスできる.
既存のアーキテクチャに基づくアプローチは,既存のDNSと根本から異なるアーキテクチャを採用しており,マイグレーションが考慮されていない.
また,ピュアのP2Pアーキテクチャおよびブロックチェーンのアーキテクチャでは,スタブリゾルバからのクエリが権威サーバとブロックチェーンをそれぞれ介することで,依然として,DNS Tunenlingとして機能し,発生を抑止するメソッドではない.
