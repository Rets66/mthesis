\section{評価}
\label{sec:evaluation}
%評価基準
%Fialure Resilience, Performance(Latency, Misconfiguration, Load Balance)
%サービスとして要求されるもの
% Resolution response time
% Resolution Accuracy
% Resolution guarantee
% Resolution fairness
%
%システムとして要求されるもの
% スケーラビリティ
% マネージビリティ
% Deployed System
% Fault Torrelance
%
%セキュリティ
% アクセシビリティ
% Aurhontication
% Data Confidentialy
% プライバシー保護
% ロバストネス
% ネットワークプライバシー
本章では,提案システムのプロトタイプ上でのトンネリング実装の動作に基づき,権威サーバにおける機能の分離と名前空間をフラットにするメソッドがトンネリング抑止に機能することを明らかにする.
また,提案システムの特性に関して,既存のDNSとの比較に基づき評価し,その特性を
トラフィック量の緩和と名前解決を高速化する有意性を示すことを説明する.

システムのプロトタイプには,Python3を用いて,フルサービスリゾルバとマネージャ,プロバイダを実装した.
DNSトンネリングに対する抑止効果については,第~\ref{sec:eval-tunnel}節で述べ,システムの特性については,第~\ref{sec:eval-feature}節でそれぞれ詳細に述べる.


%はじめに,提案システムのフルサービスリゾルバが満たす機能の要素は,以下の通りである.
%\begin{itemize}
% \item スタブリゾルバからの名前解決クエリに対して,コンテンツIDとドメイン名IDを算出する機能
% \item コンテンツIDに基づき,マネージャを探索する機能
% \item スタブリゾルバからの名前解決クエリをSORESのフォーマットに変換させ,マネージャに転送する機能
% \item マネージャからの応答パケットをDNSのフォーマットに変換させ,送信元のスタブリゾルバに転送する機能
%\end{itemize}
%
%以上の機能を提供するサービスのアルゴリズムを以下のように設計した.
%
%%client.addr = address
%client.port = port
%client.data = payload
%clinet.data.header
%client.data.qname
%client.data.rtype
%hash.sha3_224()
%hash.md5()
%map.range.start : list
%map.range.end : list
%map.addr : list
%map.name : list
%zone_x


\begin{algorithm}[!h]
 \caption{スタブリゾルバからのクエリ処理}
 \label{algo:full-service}
  \SetKwProg{Fn}{}{\string:}{}
  \SetKwFunction{Calc}{calculate\_id}
  \SetKwFunction{Find}{find\_manager}
  \SetKwFunction{Query}{query\_content}
  \SetKwFunction{Transfer}{transfer\_answer}
 $qname \leftarrow client.data.qname$\;
 $rtype \leftarrow client.data.rtype$\;
 $start \leftarrow map.range.start$\;
 $end \leftarrow map.range.end$\;
 $\vspace{-0.3cm}$\;

 Calculate the Content ID and the Domain ID\;
 \Fn{\Calc{qname, rtype}}{
   $content\_id \leftarrow hash.sha3\_224(qname+rtype)$\;
   $domain\_id \leftarrow hash.md5(qname) / 2$\;
   $return \ content\_id,\ domain\_id$
 }


 $\vspace{-0.3cm}$\;
 Find a manager who has zone includes the Content ID\;
 \Fn{\Find{start, end}}{
   \For {$i,\ j\ \textbf{in}\ map.range.start,\ map.range.end$} {
     \If {$i \leq content\_id \leq j$} {
       $p \leftarrow map.range.start.index(i)$\;
       $manager\_addr \leftarrow map.addr[p]$\;
       $break$\;
     }
   }
 }
 $\vspace{-0.3cm}$\;

 Query the content to the manager\;
 $answer \leftarrow query(manager\_addr,\ content\_id,\ domain\_id)$\;
 $\vspace{-0.3cm}$\;

 Transfer the answer to Client\;
 $transfer((client.addr,\ client.port),$\;
 $\qquad\qquad client.data.qname,\ answer.header.rcode,\ answer.rdata)$\;
\end{algorithm}

%%このアルゴリズムの説明は,本来提案手法の章で動作説明中に述べるべきであろう.
%
%
%
%% フルサービスリゾルバとプロバイダをいかにして判別するのか
%
%\begin{description}
% \item[マネージャ]\mbox{}
%  \begin{itemize}
%   \item プロバイダからのコンテンツストアリクエストに対して,コンテンツIDを探索し担当マネージャノードにコンテンツ保持をリクエストする機能
%   \item コンテンツを管理するためのデータベースを保持する機能
%   \item フルリゾルバからの名前解決クエリに対して,そのコンテンツIDに一致するコンテンツを探索し応答する機能
%  \end{itemize}
% \item [プロバイダ]\mbox{}
%  \begin{itemize}
%   \item コンテンツに対して操作する機能
%  \end{itemize}
%\end{description}

\subsection{DNSトンネリング}
\label{sec:eval-tunnel}
\subsubsection{検証環境}
DNSトンネリング抑止機能を評価するために用いるトンネリング実装は,〜を用いる,
検証環境には,Dockerを使用し,コンテナ同士の接続によってDNSのチェーンを実現する.
その概略図は,以下の通りである.

%提案システム上で,iodineを動作させた際のパケットキャプチャーした様子を説明する
\subsubsection{DNS Exfiltration}
本項では,提案システムのDNS Exfiltrationへの機能性について評価した結果を述べる.
評価には,実装したシステムを
\subsubsection{DNS Infiltration}


\subsection{要素ごとの特性}
\label{sec:eval-feature}
本節では,DNSとの比較評価に基づいた提案手法の特性について説明する.
提案システムと既存システムの名前解決における差分には,表~\ref{tab:diff_sores_dns}のようなものが予想される.
\begin{table}[htb]
 \centering
  \begin{tabular}{ccc}
    \toprule
     & \textbf{DNS} & \textbf{SORES} \\
    \midrule
    \textbf{ドメイン長} & \begin{tabular}{c}変長\\(最大253byte)\end{tabular} & \begin{tabular}{c}固定長(72byte)\\(コンテンツID(56) \& \\ ドメインID(16))\end{tabular} \\ \hline
    \textbf{\begin{tabular}{c}トランザクション\\試行回数\end{tabular}} & ラベル数回 & 1 \\ \hline
    \textbf{RTT} & \begin{tabular}{c}全ての権威サーバ\\とのRTT総和\end{tabular} & マネージャとのRTTのみ  \\ \hline
		 \textbf{ゾーンファイル} & ファイル & インメモリデータベース\\ \hline
		 \textbf{その他} & & \begin{tabular}{l}・ハッシュ計算の発生\\・マネージャ探索\end{tabular}\\
    \bottomrule
  \end{tabular}
 \caption{DNSとSORESの特性比較}
 \label{tab:diff_sores_dns}
\end{table}



\begin{description}
 \setlength{\itemsep}{0pt}
 \setlength{\leftskip}{1.0cm}
 \item[RTT:] 名前解決の速度
 \item[トラフィック量:] ネットワーク全体への影響度
\end{description}

上記の特性に関する評価にあたり,従来のDNSトラフィックのデータセットと提案システムにおける名前解決時のトラフィックデータセットを用意した.
これらデータセットについて,統計アプローチに基づいて,評価を行った結果を示す.


SORESでは,スタブリゾルバからの名前解決クエリのうち,そのヘッダー情報に基づき識別子を算出することで,この識別子に関連づけられたレコード情報を操作する仕組みに基づき名前解決機能が動作する.
DNSでは,ゾーンをドメインごとに分割し,フルサービスリゾルバはルート権威サーバから目的の権威サーバに向かって再帰的に問い合わせ,権威サーバのアドレスを解決しながら,最終的にコンテンツを保持する権威サーバからコンテンツを取得する.
既存のDNSとSORESの特性を表~\ref{tab:diff_sores_dns}に示す.


から識別子に基づき,コンテンツを保持するサーバに名前解決リクエストを投げる.
SORESは,フルサービスリゾルバにてスタブリゾルバから問い合わせられた名前解決命令に基づきレコード情報の識別子を算出する.
この識別子に基づき,フルサービスリゾルバからは,1ホップでコンテンツを保持するサーバに問い合わせる仕組みで動作する.
\subsubsection{RTT(Round Trip Time)}
本項では,SORESが識別子から一意にレコード情報にアクセス可能である点について,DNSとの比較評価を行う.
SORESの名前解決では,
従来のDNSでは,ラベルごとにゾーンが移譲されている場合,レコード情報を保持するノードまでのホップ数はラベル数nに比例する.
それに対して,SORESでは,識別子から一意にレコード情報を保持するマネージャノードを特定できるので,常にホップ数は2である.
このため,既存の名前解決システムより速度の向上が期待される.
% 実装の必要はない
% 既存のDNSにおけるラウンドトリップのうち,再帰問い合わせの最後の権威サーバRTTがSORESのフルリゾルバとマネージャのそれになる.
% 遠いものと近いもののRTTを用意する必要がある
\subsubsection{パケットサイズ}
% 実装する必要はない.理論で評価できる
% あえてするとすれば,フルサービスリゾルバとマネージャ間通信
% DDoSへの影響については,リクエストとレスポンスパケットのサイズからアンプ率に着目する
% アンプ率は,既存のDNSと基本的に変わらない.DNSSECがあるかないかについて議論するくらい
SORESでは,56byteを固定長とするコンテンツIDをシンボルとすることによって,レコード情報にアクセスする.
この仕組みの影響で,SORESのパケットは従来のパケットと比較して肥大する特性がある.
このため,送信元を目的ホストと偽装することで目的ホストの計算リソースを圧迫するDDoS攻撃に対して,脅威を高める可能性が予想される.
\subsubsection{トラフィック量}
% SORESについては,実装の必要性はない
% ただし,DNSについてトラフィックデータを収集する必要がある
% 必要なデータ
% * 一定期間のSORESトラフィック
% * 同一期間のDNSトラフィック
% 比較項目
% * 平均したドメイン長と
本項では,提案手法におけるトラフィックについて評価する.
評価において,
SORESでは,シンボル志向の名前解決メソッドによって,既存の再帰問い合わせによるメソッドよりも少ないトラフィックに抑えることが期待される.
SORESでは,クエリ数とトラフィック量は比例関係にある.
他方で,新たにマネージャ間通信という従来にはないトラフィックが発生する.
本項では,これトラフィックがネットワーク全体にどの程度影響を及ぼすのかについて評価する.
% マネージャが攻撃された場合の影響は,世界規模に影響する.
% 従来のドメインごとにゾーンが分離している設計と違い,一つのゾーンには様々な組織のドメインが管理されている.
% このゾーンを全て書き換えることができることは脅威である.

