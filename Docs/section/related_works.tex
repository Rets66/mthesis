\section{関連研究}
\label{sec:related-works}
本章では,はじめに既存のDNS Tunnelingに対するアプローチとして提案されている検知アプローチを取り上げ,現在の検知に基づく対策の課題として,Low Throughput手法とSlowな転送手法というバイパス手法に対処できないことを明らかにする.
次に,これまでに提案されてきたP2Pベースの名前解決システムを説明し,提案手法との違いを示す.
%また,新しいアーキテクチャを導入するとき 既存のシステムとのマイグレーションを考慮する必要がある点について,本提案手法がマイグレーションを考慮している設計である点について
%最後に,既存の検知手法および次世代名前解決システムの課題から,既存のシステムに迎合しながらDNS Tunnelingを緩和する名前解決システムの必要性を明らかにする.

\subsection{特徴量に基づく検知手法}
\subsubsection{閾値推定}
\subsubsection{機械学習に基づくモデル}
%\subsubsection{パターンマッチング}
%\subsubsection{同一ドメインあたりのクエリ頻度}
%\subsubsection{Qnameにおける文字列分布}
%\subsubsection{Qnameにおける長さとエントロピー}
%\subsubsection{課題 : Low ThroughputなTunnelingに対する検知手法}
DNS Tunnelingメソッドを使用した時のDNSクエリは,\ref{sec:dns-tunnel}で述べたような特徴が現れる性質がある.
この性質に基づき,これまでに多数の検知手法が提案されてきた.

Bornら~\cite{born}は,

%\subsection{悪性DNS検知に関する研究}
\subsection{新しいDNSアーキテクチャ}
\subsubsection{Peer-to-Peerネットワークを利用した名前解決システム}
これまでに,DNSにおける〜の課題に対して,P2Pに基づいた名前解決システムは数多く提案されてきた.
\subsubsection{Blockchainを利用した名前解決システム}
\subsubsection{フラットな名前空間に基づく名前解決システム}
\subsection{課題}
\label{sec:issue-past-works}
