\section{結論}
% 機能を分離させることによって,クエリを用いてデータを転送する仕組みを無効化することを実現
% 他の名前解決システムと違い,クライアントサイドへの変更点が少ないためマイグレーションが期待される
本論文では,DNSトンネリングを抑止する名前解決システムを提案した.
DNSトンネリングが動作する仕組みについて分析し,修正する対象として,スタブリゾルバから権威サーバまでクエリパケットが転送される仕組みに着目した.
提案システムDNS-TDは,権威サーバのコンテンツを保持およびクライアントからの問い合わせに応答する機能とコンテンツを編集する機能を2つサービスノードに分割させる設計をとった.
これによって,任意のサーバを設置することでスタブリゾルバとの間でデータがやりとりされることが抑制され,DNSトンネリング通信発生の抑止に寄与する.
DNSは現在のインターネットの根幹技術であり,基盤システムとして根付いているため,大幅な主星が加えられた場合,接続障害が発生する可能性があり,マイグレーションを念頭に入れた設計が重要である.
提案システムは,フルサービスリゾルバと権威サーバにおける名前解決の仕組みのみの変更に抑えているため.現実的なマイグレーションが期待される.
評価では,実装した提案システムのプロトタイプ上で,擬似トンネリング通信を発生させるシミュレーションテストを行うことで,提案システムがトンネリング通信を抑止できることを示した.
また,提案システムの特性の評価を行い,提案システムが既存システムよりも少ないトラフィックでかつ高速な名前解決を実現できるという優位性があることを示した.

提案システムは,ドメインに依らないフラットな名前空間の範囲に基づいてゾーン分割する設計上,1つのマネージャが保持するゾーンには複数のドメインが含まれる.
そのため,1つのマネージャの障害がインターネット全体に影響が及ぶ,潜在的な脅威がある.
今後の課題は,この脅威の発生を緩和する負荷分散の仕組みを検討することである.
