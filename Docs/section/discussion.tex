\section{考察}
\label{sec:discussion}
%第~\ref{sec:evaluation}章で示す通り,提案システムにおける権威サーバの機能の分離と識別子に基づくマネージャの探索手法によって,トンネリングを抑止しながら既存の名前解決システムの基盤を踏襲できることを明らかにした.
%また,比較に基づく特性評価から,提案システムはトラフィック量の削減と名前解決の高速化が期待されることを示した.
本章では,提案システムにおける課題と有用性について考察する.

\subsection{提案システムをバイパスするトンネリング手法}
提案システムにはマネージャ以外のサービスノードにサーバ機能がないため,マネージャに脅威が及ばない限りトンネリングの機能を抑止することができる.
%提案システムでは,権威サーバの機能を分離させ,コンテンツを提供するサーバをユーザのアクセスできる範囲から離すことによって,トンネリングの発生を抑止させている.
しかし,何らかの手法によってマネージャが攻撃される場合,マネージャを通じてトンネリング通信される潜在的な可能性はある.
そのような攻撃手法には,細工したDNSクエリを用いることが考えられる.
クエリパケットを通じてマネージャの権限を取得する攻撃には,クエリを分解する機能にエスケープ処理を施す実装するにすることによって対処できると考えられる.

\subsection{マイグレーション}
基盤システムへの修正において,マイグレーションの実現性は重要である.
既存システムから提案システムに移行させるプロセスでは,はじめに識別子の名前空間に関してどのマネージャがどの範囲をゾーンとするのかを決める.
マネージャが保持するドメインの候補となるのは,一部のドメインを除いたgTLDである.
2020年1月時点で,1514個のドメインがgTLDとして登録されている.
マネージャが保持するドメイン情報に基づき,そのドメインに関連したドメインを保持するプロバイダが連結されるため,マネージャのドメインはドメイン全体を分類し易い一般的なドメインであることが望ましい.
そのため,一部の既存ドメインはプロバイダが保持するドメインに位置づけ,関連するマネージャドメインに連結させる処理を施していく.

次に,データベースを保持するマネージャを動作させ,既存システムにおける権威サーバによって管理されていたドメイン名とレコーとタイプの組みの情報について算出される識別子に基づき,その識別子をゾーンとするマネージャにコンテンツ情報をフィードしていく.
マネージャにコンテンツ情報が貯められる状態になると,識別子をQnameに含めることで希望の名前解決サービスを提供する基盤が整う.
上記のマネージャへのコンテンツ情報のフィード処理と並行しながら,フルサービスリゾルバに識別子を導出しQnameに識別子を含める機能をBindやUnboundをはじめとするフルサービスリゾルバのソフトウェアに実装する.
これらの手続きを実現することによって,既存システムから提案システムの名前解決基盤に移行させることができる.


\subsection{ハッシュ関数危殆化の影響とシステムの継続性}
%ハッシュ関数の安全性は,計算時間の長さに基づき評価される.
計算リソースの向上や解析アルゴリズムの効率化などによって,使用するハッシュアルゴリズムが危殆化する可能性がある.
提案システムでは,ハッシュ関数の原像計算困難性に基づき,フルサービスリゾルバとサーバ間におけるクエリの機密性を確保している.
ハッシュ関数が危殆化した場合,クエリパケット内の識別子から元のメッセージに復元されるというプライバシー侵害の脅威に発展する.
このように提案システムの目的であるトンネリング抑止の機能に対しては影響はないものの,プライバシーに影響が及ぶ課題がある.
フルサービスリゾルバとマネージャ間通信におけるトラフィックを暗号化させる仕組みが今後の課題である.
%提案システムにおけるハッシュ関数の利用は,元のメッセージから相関のないダイジェストが算出される性質を主として利用している.

%\subsection{DNSSECの位置づけ}

\subsection{範囲に基づくゾーン分割に起因する脅威}
提案システムでは,マネージャというサービスノードへの信頼に基づいて,ドメイン名とレコードタイプのペアに識別子を付与し,複数のマネージャが識別子が帰属する名前空間の範囲を協調的に管理することでコンテンツが管理される.
この設計には,マネージャに障害が発生した際には極めて広範囲に影響が及ぶ潜在的な脅威がある.
それには,ユーザとの単一接続点となるマネージャへのDoS攻撃\footnote{DoS(Denial of Service)攻撃: サービス不全攻撃.サーバに大量の問い合わせることによって,サーバの計算リソースを意図的に消費されサービス提供を不全にさせる攻撃手法.}のアプローチが攻撃手法として考えられる.
DoS攻撃に対して,既存システムではラウンドロビンによる負荷分散の手法が考えられるが,提案システムでは検討されていない.
提案システムにおけるマネージャの負荷分散機能は,今後の課題である.
%自動ロードバランシング


%\subsection{マネージャドメインの選抜とプロバイダの配属}
% マネージャノードが悪意のあるユーザによって攻撃された際については,議論する必要があるのか
% - プロバイダからの問い合わせをパースする際にセキュリティ機構を備えることによって,対策を講じる必要性がある
%\subsection{マネージャノードの最適な数}
% 提案システムでは,全てのネットワークが統一的な名前解決基盤を使用することになるため,国家レベルで検閲が実施されている環境には極めて大きな影響を及ぼす
