\section{関連研究}
\label{sec:related-works}
本章では,はじめにDNS Tunnelingに対する既存の対策手法の概要と課題を説明する.
次に,従来とは異なるアーキテクチャを採用した名前解決システムに関する概要と提案手法との違いについて説明する.
%最後に,既存の検知手法および次世代名前解決システムの課題から,既存のシステムに迎合しながらDNS Tunnelingを緩和する名前解決システムの必要性を明らかにする.

\subsection{特徴量に基づく検知手法}
本節では,DNS Tunnelingに対する対策手法について説明する.
\subsubsection{DNS Tunnelingにおける特徴量}
DNS Tunnelingの手法を用いる場合,いくつかの特徴が現れることがある.
Bornら~\cite{born}は,DNS Tunnelingにおけるクエリ内のドメイン名に出現する文字の出現頻度を着目
することで,正規のドメイン名とDNS Exfiltrationメソッドによって生成されたドメイン名とを分類のに有用であることを明らかにした(2010).
著者らは,正規のドメインであれば英語のような自然言語における文字の出現頻度と高い相関があることに加えて,エントロピーと文字列分布に高い相関があることも明らかにした.


\subsubsection{閾値推定}
\subsubsection{機械学習に基づくモデル}
%\subsubsection{パターンマッチング}
%\subsubsection{同一ドメインあたりのクエリ頻度}
%\subsubsection{Qnameにおける文字列分布}
%\subsubsection{Qnameにおける長さとエントロピー}
%\subsubsection{課題 : Low ThroughputなTunnelingに対する検知手法}
DNS Tunnelingメソッドを使用した時のDNSクエリは,第~\ref{sec:dns-tunneling}項で述べるような特性が出現する.
この性質に基づき,これまでに多数の検知手法が提案されてきた.


%\subsection{悪性DNS検知に関する研究}
\subsection{ポスト名前解決システム}
%また,新しいアーキテクチャを導入するとき 既存のシステムとのマイグレーションを考慮する必要がある点について,本提案手法がマイグレーションを考慮している設計である点について
\subsubsection{P2Pネットワークを利用した名前解決システム}
これまでに,DNSにおける〜の課題に対して,P2Pに基づいた名前解決システムは数多く提案されてきた.
\subsubsection{Blockchainを利用した名前解決システム}
%\subsubsection{フラットな名前空間に基づく名前解決システム}
\subsection{課題}
\label{sec:issue-past-works}
本節では,先行研究および新しいアーキテクチャに基づく名前解決システムにおけるDNS Tunnelingへの課題を示す.
(大筋)検知に基づく手法は,誤検知が避けられない.ペイロードアナリシスに対しては,一回あたりの転送量を調整することでバイパスすることができる.
分析の対象がトラフィックの場合は,頻度を長期間に延長することで,ログファイルの肥大によるストレージの過去のログファイルとの分析コストを重くなり分析の隙間をバイパスできる.
既存のアーキテクチャに基づくアプローチは,既存のDNSと根本から異なるアーキテクチャを採用しており,マイグレーションが考慮されていない.
また,ピュアのP2Pアーキテクチャおよびブロックチェーンのアーキテクチャでは,スタブリゾルバからのクエリが権威サーバとブロックチェーンをそれぞれ介することで,依然として,DNS Tunenlingとして機能し,発生を抑止するメソッドではない.
