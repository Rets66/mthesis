%評価では,まずは目的とするDNSトンネリングの発生を抑止できるのかに焦点を当てるべき
%すなわち,DNSトンネリングのクエリを実際に問い合わせた時のネットワーク動作を評価するなど
% 次に,提案システムが既存システムと比較して異なる特性がある,または課題になる特性が明白である場合,どの程度それらが問題になるのかについて実際に評価すればいい
%評価基準
%Fialure Resilience, Performance(Latency, Misconfiguration, Load Balance)
%サービスとして要求されるもの
% Resolution response time
% Resolution Accuracy
% Resolution guarantee
% Resolution fairness
\section{評価}
\label{sec:evaluation}
提案システムでは,DNSトンネリング抑止の目的に対して,フラットな名前空間とDHTの仕組みを採用している.
このアプローチは,名前解決のパフォーマンスにおいて,コンテンツの識別子に用いられるメッセージダイジェストの算出に伴うメモリ消費と問い合わせノードの探索の処理がオーバーヘッドになることが予想される.
本章では,DNSトンネリング抑止としての提案システム設計の有用性と提案システムの特性について評価した結果を述べる.
%フラットな名前空間とDHTに基づき分散的な管理する提案システムが,DNSトンネリング抑止に対して有効であることを明らかにする.
評価では,提案システムのプロトタイプを実装することで,シミュレーションに基づいて評価を行った.

また,既存システムとの比較に基づいたシミュレーションテストにより,名前解決速度・トラフィック量などの特性を明らかにする.

トンネリング抑止の機能評価テストでは,フルサービスリゾルバとマネージャおよびプロバイダをPython3を用いて実装し,それらサービスをDockerのコンテナとして動作させることによって,提案システムにおける名前解決環境を再現したプロトタイプを用いた.
また,DNSトンネリングの通信には,ランダムなドメイン名をdigコマンドを用いることによって再現した.
プロトタイプ上におけるdigコマンドによる擬似トンネリング通信について,プロバイダにデータが転送されないことを確認する.
特性評価では,ランダムな名前解決クエリを用意し,既存のDNSと提案システムの両方で問い合わせた際のトラフィックと名前解決の速度について,統計的に評価を実施する.

\subsection{プロトタイプ実装}
検証環境には,Dockerを使用し,コンテナ同士の接続によってDNSのチェーンを実現する.
その概略図は,以下の通りである.

\subsection{名前解決の機能}
\label{sec:eval-tunnel}
提案システムDNS-TDにおける名前解決の動作を明らかにするために,本項では実際にシステムをDocker環境上で動作させたシミュレーションの結果を説明する.
提案システムのDNS Exfiltrationへの機能性について評価した結果を述べる.

%実際にコンテンツが登録されている場合とコンテンツ情報が含まれていない場合(DNSクエリ)

%擬似DNSトンネリングとするドメイン名については,過去の文字列分布が均等分布になることに基づいてランダムな関数で作成したこと理論について説明すること
%提案システム上で,iodineを動作させた際のパケットキャプチャーした様子を説明する

\begin{figure}[p]
 \centering
 \includegraphics[width=14.5cm]{figure/stab-fullservice.png}
 \vspace{-1cm}
 \caption{スタブリゾルバからフルサービスリゾルバにおける通信}
 \label{fig:fullservice-manager}
 \vspace{1cm}
 \includegraphics[width=14.5cm]{figure/fullresolver-manager.png}
 \vspace{-1cm}
 \caption{フルサービスリゾルバからマネージャにおける通信}
 \label{fig:fullservice-manager}
\end{figure}

\newpage
\subsection{要素ごとの特性}
\label{sec:eval-feature}
本節では,DNSとの比較に基づき提案システムにおける特性について評価する.
表~\cite{tab:diff_feature}で示す通り,既存システムと提案システムでは以下のような特性の違いがある.
本節では,これら特性の違いを踏まえて,DNS-TDにおける名前解決のパフォーマンスとトラフィック量に明らかにする.
\begin{table}[htb]
 \caption{DNSとDNS-TDの特性比較}
 \centering
  \begin{tabular}{lll}
    \toprule
		 & \multicolumn{1}{c}{\textbf{DNS}} & \multicolumn{1}{c}{\textbf{DNS-TD}} \\
    \midrule
		\begin{tabular}{l}\textbf{Qnameサイズ}\\\textbf{(bytes)}\end{tabular} & \begin{tabular}{l}変長\\(最大253)\end{tabular} & \begin{tabular}{l}固定長(85)\\(コンテンツID(56).(1)\\ドメインID(28))\end{tabular} \\ \hline
		\begin{tabular}{l}\textbf{問い合わせ回数}\end{tabular} &\begin{tabular}{l}委譲された回数\end{tabular} & \begin{tabular}{r}1\end{tabular} \\ \hline
		\begin{tabular}{l}\textbf{RTT}\end{tabular} & \begin{tabular}{l}全ての権威サーバ\\とのRTT総和\end{tabular} & \begin{tabular}{l}マネージャとのRTTのみ\end{tabular}  \\ \hline
		\begin{tabular}{l}\textbf{ゾーン管理}\end{tabular} & \begin{tabular}{l}ファイル\end{tabular} & \begin{tabular}{l}データベース\end{tabular}\\ \hline
		\begin{tabular}{l}\textbf{オーバーヘッド}\end{tabular} & \begin{tabular}{l}再帰問い合わせ\end{tabular} & \begin{tabular}{l}・コンテンツIDの計算\\・ドメインIDの計算\\・マネージャ探索処理\end{tabular}\\
    \bottomrule
  \end{tabular}
 \label{tab:diff_dns-td_dns}
\end{table}


\subsubsection{オーバヘッド}
DNS-TDにおける名前解決処理では,まずはじめにフルサービスリゾルバがマネージャノードを決定する必要がある.
マネージャノードは,スタブリゾルバから問い合わせられたドメイン名とレコードタイプから算出されるコンテンツIDに基づいて決まる.
このため,フルサービスリゾルバは,スタブリゾルバからのクエリをサーバに転送する処理に加えて,コンテンツIDの算出処理を実行する必要がある.
コンテンツIDは,ドメイン名とレコードタイプの和をメッセージとするハッシュ関数から算出されるメッセージダイジェストである.
以上のことから,DNS-TDでは,コンテンツIDの算出に伴う処理が名前解決におけるパフォーマンスのオーバーヘッドになることが予想される.
そこで,本項では,名前解決におけるコンテンツIDの算出処理にかかるメモリおよび時間のオーバーヘッドについて調べた.

DNS-TDで使用されるハッシュ関数には,224bitの名前空間をもつsha3というハッシュ関数が用いられている.
%DNSは,現在のインターネットの根幹に位置する技術であるため,DNSトラフィックはインターネット全体に大きく影響する.


\subsubsection{再帰問い合わせの有無に関する名前解決の速度}
\label{sec:resolution_speed}
DNS-TDにおけるゾーンは,ソートされたハッシュ名前空間の範囲によって分割されているため,コンテンツIDから担当のコンテンツサーバを一意に求めることができる.
他方で,既存システムのDNSでは,ドメイン名の階層構造と委譲の仕組み基づいてゾーンが分割されているため,コンテンツを保持する権威サーバにはルート権威サーバから再帰的に問い合わせる必要がある.
このため,既存システムでは,ドメイン名においてドメインが委譲された分だけ権威サーバに問い合わせる必要がある.
以上のように,DNS-TDとDNSの名前解決処理には,サーバに問い合わせる数の違いがある.
そこで,本項では,サーバへの問い合わせ数の違いが名前解決の速度にどの程度影響を及ぼすのかについて調べた.

\subsubsection{トラフィック量}
第~\ref{sec:resolution_speed}項で述べるように,DNSではサーバに再帰的に問い合わせることで名前が解決されるのに対して,DNS-TDではサーバへの問い合わせは一度で済む.
このため,DNS-TDでは,DNSと比べて少ないトラフィック頻度となることが予想される.
しかし,パケットサイズに関して,DNSではドメイン名から
他方で,DNS-TDでは固定長のドメイン名がサーバに問い合わせられるのに対して,

DNSでは,ゾーンをドメインごとに分割し,フルサービスリゾルバはルート権威サーバから目的の権威サーバに向かって再帰的に問い合わせ,権威サーバのアドレスを解決しながら,最終的にコンテンツを保持する権威サーバからコンテンツを取得する.

\subsubsection{RTT(Round Trip Time)}
本項では,DNS-TDが識別子から一意にレコード情報にアクセス可能である点について,DNSとの比較評価を行う.
DNS-TDの名前解決では,
従来のDNSでは,ラベルごとにゾーンが移譲されている場合,レコード情報を保持するノードまでのホップ数はラベル数nに比例する.
それに対して,DNS-TDでは,識別子から一意にレコード情報を保持するマネージャノードを特定できるので,常にホップ数は2である.
このため,既存の名前解決システムより速度の向上が期待される.
% 実装の必要はない
% 既存のDNSにおけるラウンドトリップのうち,再帰問い合わせの最後の権威サーバRTTがdns-tdのフルリゾルバとマネージャのそれになる.
% 遠いものと近いもののRTTを用意する必要がある
\subsubsection{パケットサイズ}
% 実装する必要はない.理論で評価できる
% あえてするとすれば,フルサービスリゾルバとマネージャ間通信
% DDoSへの影響については,リクエストとレスポンスパケットのサイズからアンプ率に着目する
% アンプ率は,既存のDNSと基本的に変わらない.DNSSECがあるかないかについて議論するくらい
%既存システムにおけるドメイン長の平均
DNS-TDでは,56byteを固定長とするコンテンツIDをシンボルとすることによって,レコード情報にアクセスする.
この仕組みの影響で,DNS-TDのパケットは従来のパケットと比較して肥大する特性がある.
このため,送信元を目的ホストと偽装することで目的ホストの計算リソースを圧迫するDDoS攻撃に対して,脅威を高める可能性が予想される.
\subsubsection{トラフィック量}
% DNS-TDについては,実装の必要性はない
% ただし,DNSについてトラフィックデータを収集する必要がある
% 必要なデータ
% * 一定期間のDNS-TDトラフィック
% * 同一期間のDNSトラフィック
% 比較項目
% * 平均したドメイン長と
本項では,提案手法におけるトラフィックについて評価する.
DNS-TDでは,シンボル志向の名前解決メソッドによって,既存の再帰問い合わせによるメソッドよりも少ないトラフィックに抑えることが期待される.
DNS-TDでは,クエリ数とトラフィック量は比例関係にある.
他方で,新たにマネージャ間通信という従来にはないトラフィックが発生する.
本項では,これトラフィックがネットワーク全体にどの程度影響を及ぼすのかについて評価する.
% マネージャが攻撃された場合の影響は,世界規模に影響する.
% 従来のドメインごとにゾーンが分離している設計と違い,一つのゾーンには様々な組織のドメインが管理されている.
% このゾーンを全て書き換えることができることは脅威である.

% 名前解決にかかる平均的なホップ数の比較
% 既存システムが2-3とされている
%クエリminimizationがどの程度普及しているのかによって,計算方法が随分と異なることが予想される
