\section{評価}
\subsection{実装}
\subsection{実験環境}
\subsection{DNS Tunnelingに対する理論評価}
\subsubsection{評価要素 : 対Exfiltration}
\subsubsection{評価要素 : 対Infiltration}
\subsection{シミュレーション実験に基づく特性評価}
\subsubsection{評価要素 : パケットサイズ}
REFRESでは,224bitを固定長とするコンテンツIDをシンボルとすることによって,レコード情報にアクセスする.
この仕組みの影響で,REFRESのパケットは従来のパケットと比較して肥大する特性がある.
このため,送信元を目的ホストと偽装することで目的ホストの計算リソースを圧迫するDDoS攻撃に対して,脅威を高める可能性が予想される.
\subsubsection{評価要素 : RTT(Round Trip Time)}
従来のDNSでは,ラベルごとにゾーンが移譲されている場合,レコード情報を保持するノードまでのホップ数はラベル数nに対してO(n)になる.
対して,REFRESでは,識別子から一意にレコード情報を保持するマネージャノードを特定できるので,O(2)のホップ数となる.
このため,既存の名前解決システムより速度の向上が期待される.
\subsubsection{評価要素 : トラフィック量}
REFRESでは,シンボル志向の名前解決メソッドによって,既存の再帰問い合わせによるメソッドよりも少ないトラフィックに抑えることが期待される.
他方で,新たにマネージャ間通信という従来にはないトラフィックが発生する.
本項では,これトラフィックがネットワーク全体にどの程度影響を及ぼすのかについて評価する.
