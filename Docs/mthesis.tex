\documentclass[12pt]{jarticle} % Japanese
%\documentclass[12pt]{article} % English
% if there are problems in the above regarding fonts, use this
% \documentclass[UTF8]{ctexart}



%package
\usepackage{amsmath, newtxmath}
\usepackage[utf8]{inputenc}
%\usepackage{utf}
\usepackage{naist-jmthesis} %Japanese
%\usepackage{naist-mthesis} %English
\usepackage{graphicx}
\usepackage[dvipdfmx]{hyperref}
\usepackage{pxjahyper} %Required pxjahyper.sty
\usepackage{xcolor}
\usepackage{pxjahyper} %TOC文字化け対策



%definition
\definecolor{purple}{RGB}{98, 114, 164}
\hypersetup{
 colorlinks=true,
 linkcolor=black,
 citecolor=purple,
 urlcolor=purple,
 pdfborder={0, 0, 1},
 linktoc=all
}



% Page style
\pagestyle{final}       % Camera-Ready
%\pagestyle{draft}      % Draft
\lang{Japanese} % Japanese
%\lang{English} % English
% Student Number
\studentnumber{1811147}
\doctitle{\mastersthesis}       % 修士論文
\major{\engineering}    % 工学



% 日本語題目 (in LaTeX)
%\title{再帰問い合わせ名前解決へのハッシュ関数を用いたDNS Exfiltration緩和策の提案}
\title{DNS Exfiltration対策を目的としたスーパーノード型P2Pネットワークに基づく名前解決システムの提案}
% 日本語題目 (in plain text)
%   注: (in LaTeX)と同じ場合は指定する必要なし。
%       この情報は修士論文/課題研究には現れませんが、管理のために必要です。
%\ptitle{再帰問い合わせ名前解決へのハッシュ関数を用いたDNS Exfiltration緩和策の提案}



% 英語題目 (in LaTeX)
%\etitle{Proposal for Mitigation of DNS Exfiltraion using Hash Function to Recursive Name Resolution}
\etitle{Proposal for Name Resolution System based on Supernode in P2P Networks against DNS Exfiltration}
% 英語題目 (in plain text)
%   注: (in LaTeX)と同じ場合は指定する必要なし。
%       この情報は修士論文/課題研究には現れませんが、管理のために必要です。
%\eptitle{Theoretical Studies on Low-Speed Calculation Algorithms of pi \\
%Utilizing the Sun and the Moon}
%\eptitle{Proposal for Mitigation of DNS Exfiltraion using Hash Function to Recursive Name Resolution}



% 日本語氏名 (in LaTeX)
%   (姓と名の間に空白を入れて下さい)
\author{高須賀 昌烈}
%\pauthor{}
%   (first name, last name の順に記入し、先頭文字のみを大文字にする。)
\eauthor{Shoretsu Takasuka}
% 別の例: \eauthor{Kurt G\"{o}del}
%\epauthor{}



% 論文提出年月日
\syear{2020}
\smonth{3}
\sday{15}



% 専攻の選択
%\department{\infproc}  % 情報処理学
%\department{\infsys}    % 情報システム学
%\department{\bioinf}   % 情報生命科学
\department{\infsci}    % 情報科学



% 審査委員(日本語)
%   (姓と名、名と称号の間に空白を入れて下さい)
%5人以上の場合,5人目以降は\addcmembers を使って宣言する。
%最大で合わせて8人まで宣言可能。
%主指導教員、副指導教員を明記する。両指導教員以外は委員。
%学外審査委員は、大学名を明記する
% 4人の場合
\cmembers{門林 雄基 教授}{(主指導教員)}
         {笠原 正治 教授}{(副指導教員)}
         {林 優一 教授}{(副指導教員)}
         {妙中 雄三 准教授}{(副指導教員)}
% 3人の場合
%\cmembers{○○ ○○ 教授}{(主指導教員)}
%         {○○ ○○ 教授}{(副指導教員)}
%         {○○ ○○ 准教授}{(副指導教員)}
%         {}{}
% 2人の場合
%\cmembers{○○ ○○ 教授}{(主指導教員)}
%         {○○ ○○ 教授}{(副指導教員)}
%          {}{}
%          {}{}



% 審査委員(英語)
%     (first name, last name の順に記入し、先頭文字のみを大文字にする。
%       first name と last name の間に空白、
%       last name と 称号の間にカンマと空白を入れて下さい。)
% 5人以上の場合,5人目以降は\eaddcmembers を使って宣言する
% Supervisor, Co-supervisor, and Member must be specified.
% 4人の場合
\ecmembers{Professor Youki Kadobayashi}{(Supervisor)}
          {Professor Shoji Kasahara}{(Co-supervisor)}
          {Professor Yu-ichi Hayashi}{(Co-supervisor)}
          {Associate Professor Yuzo Taenaka}{(Co-supervisor)}
% 3人の場合
%\ecmembers{Professor xx xx}{(Supervisor)}
%          {Professor xx xx}{(Co-supervisor)}
%          {Associate Professor xx xx}{(Co-supervisor)}
%          {}{}
% 2人の場合
% \ecmembers{Professor xx xx}{(Supervisor)}
%           {Professor xx xx}{(Co-supervisor)}
%           {}{}
%           {}{}
% キーワード5〜6個 (in LaTeX)
%\keywords{$\pi$, 天文学, 数学, 計算機, アルゴリズム}



% ===================キーワード===================
\keywords{ネットワークセキュリティ,ドメインネームシステム,秘匿通信,分散ハッシュテーブル,スーパーノード型ピアツーピア}
% キーワード5〜6個 (in plain text)
%   注: (in LaTeX)と同じ場合は記入する必要なし。
%       この情報は修士論文/課題研究には現れませんが、管理のために必要です。
%\pkeywords{pi, 天文学, 数学, 計算機, アルゴリズム}
%\pkeywords{DNS Exfiltration, 秘匿通信,ハッシュ関数,再帰問い合わせ}
% 5 or 6 Keywords (in LaTeX)
%\ekeywords{$\pi$, astronomy, mathematics, computer, algorithm}



% ===================Keyword===================
\ekeywords{Network Security, Domain Name System(DNS), Covert Channel, Distributed Hash Table(DHT), Supernode in Peer-to-Peer}
% 5 or 6 Keywords (in plain text)
%   注: (in LaTeX)と同じ場合は記入する必要なし。
%       この情報は修士論文/課題研究には現れませんが、管理のために必要です。
%\epkeywords{pi, astronomy, mathematics, computer, algorithm}
%\epkeywords{DNS Exfiltration, Covert Channel, Hash Function, Recursive Name Resolution}



% ===================内容梗概===================
\abstract{
}
%   注: 行の先頭が\\で始まらないようにすること。
%   注: (in LaTeX)と同じ場合は記入する必要なし。
%       この情報は修士論文/課題研究には現れませんが、管理のために必要です。
%       改行する箇所には空白行を入れる。
%       行の先頭が\\で始まらないようにすること。
%\pabstract{
%}
% Abstract (in LaTeX)
%  注:  行の先頭が\\で始まらないようにすること。



% ===================Abstruct===================
\eabstract{
}
% Abstract (in plain text)
%   注: (in LaTeX)と同じ場合は記入する必要なし。
%       この情報は修士論文/課題研究には現れませんが、管理のために必要です。
%       改行する箇所には空白行を入れる。
%       行の先頭が\\で始まらないようにすること。
%\epabstract{
%The calculation of pi has been paid much attention since human beings
%appeared on the earth.
%This thesis presents novel low-speed algorithms to calculate
%pi utilizing the sun and the moon.
%This is a sample abstract. This is a sample abstract.
%}



% ===================表紙===================
\begin{document}
\titlepage
\cmemberspage
\firstabstract
\secondabstract



% ===================目次===================
\toc
\newpage
\listoffigures
%\newpage
\listoftables



% ===================本文===================
\newpage
\pagenumbering{arabic}
\section{序論}
\subsection{背景}
ドメインネームシステム(Domain Name System, DNS)は,ドメイン名(E.g. www.example.com)をインターネット上でのノードの住所を表すIPアドレス(E.g. 93.184.216.34)に変換する機能を担っており,DNSを通じて特定した宛先に問い合わせることでユーザはそのサービスにアクセスできている.
このように現在のインターネットの利活用において,名前解決の仕組みは極めて重要な技術の一つである.

1987年にRFC1034, RFC1035(\cite{rfc1034, rfc1035})として公開されたDNSのコンセプトは,現在もなお本質的な仕組みは変更されることなく適用されている.
しかし,性善説的な当時の設計に伴い生じた脆弱性を利用した攻撃がいくつか報告されている.
%近年では,平文であるDNSクエリを解析することでユーザがどのwebページを閲覧しようとしているのか,どこにメールを送信しようとしているのかといったユーザのプライバシーを侵害される脅威\cite{rfc7626}に関心が集まっており,HTTPSやTLSによってクエリおよび応答パケットを暗号化するDoH/DoTが盛んに議論されている.
その設計に起因する課題の内,DNSクエリのラベルおよびリソースレコード(Resource Record, RR)をデータ転送のメディアとするDNS Tunnelingがある.

DNS Tunnelingは,一般にフィルタリングされることが少ないDNSの特徴とDNSがデータ転送のメディアとして機能しているとは想像しない人の認知の隙間をついた手法であり,ファイヤー・ウォールやIDS/IPSといったセキュリティラインを突破するために使用される.
このように本来の目的とは違う方法でデータを転送する手法は,一般に秘匿通信(Covert Channel)と呼ばれる\cite{covertchannel}.
DNS Tunnelingは,秘匿通信の代表例であり,マルウェアとC2(Command \& Control)サーバとの通信の秘匿手法,または,ターゲットから取得したデータを外部に流出させるといった目的実行の手段として,実際のインシデントで広く利用されている\cite{frameworkpos, bondupdater, bernhardpos, multigrainpos, pisloader, denis, dnsmessenger, udpos}.
%2014年には,発生した大規模なクレジットカード情報流出事件\cite{},最近では2019年に発生したAPTグループ(通称,OilRig)による中東政府を標的とするサイバー攻撃のC2通信\cite{bondupdater}として実際のインシデントなどがある.
%このDNS Tunnelingのメカニズムは,スタブリゾルバからリカーシブサーバを経由し権威サーバへ問い合わせる一連の正規の仕組みに基づいており,名前解決を実現するにあたり生じる設計上の脆弱性を突いた手法である.
従来のDNS Tunnnlingに対するアプローチには,検知による手法が採用されてきた.
DNS TunnelingによるDNSクエリは,以下(\ref{eq:sample_qname})に示すように,転送量に比例して長いラベルを持ち,ラベルとしての文字列制約を満たすためのエンコーディングによって高いエントロピーを示す特徴がある.

\begin{eqnarray}
 \label{eq:sample_qname}
 \begin{aligned}
  &obqxg43xmqytcmjr.exfil.com\\
  &base32(password1111) = obqxg43xmqytcmjr
 \end{aligned}
\end{eqnarray}

また,インタラクティブなシェルなど双方向の通信をDNS Tunnelingで実現しようとする場合,時間あたりに高頻度なトラフィックが発生するという特徴が現れる.
このような特徴に基づき,パターンマッチングや機械学習,文字列分布などのメソッドを用いた検知手法が過去に多数考案されてきた\cite{born, cheng, liu, asaf, steadman, jawad}.
それら検知手法は,かなり高い精度で分類を実現しているものがあるが,DNS Tunnelingとして検知する対象としているパケットには一般に利用することができるDNS Tunnelingツールキット\cite{ozymandns, iodine, dnscat2}が使用され,それらは特に過剰な特徴量を示し,明らかに正規のDNSクエリと異なる特徴がある.
高い精度を示す従来の検知手法だが,しかし,それらを迂回する手法として,1回あたりの転送データ量を少なくすることで特徴量を減らすLow Throughputなバイパス手法,また,パケット間のインターバルを数日・数ヶ月と長期化させることでファイル肥大から一定期間しか保存されることがないログ管理の隙間を突いたSlowなTunenling手法があり,従来の検知手法では対応することが困難である.
悪意を持つユーザの視点として,1bitでも転送できることは秘匿通信として利用することができるため,転送量の少なさは軽視されるべきではない.

他方で,DNSは初めに述べたように,現在のインターネットの根幹技術として根ざしており,抜本的な改変は期待されない.
すなわち,既存のDNSによる名前解決のメカニズムに大幅な改変を加えないという制約下で,Tunnelingに対処することが現実的な最適解であると考える.


\subsection{目的}
本研究では,既存のDNSの名前解決メカニズムの大部分を流用することが一部の改変に留めながら,DNSを用いたデータ転送としての機能の排除を実現する次世代の名前解決メカニズムを提案する.
%\subsection{貢献点}
%本研究の貢献は,以下の通りである.
%\begin{itemize}
% \item 侵入通信を目的とするDNS Tunnelingに対するリアルタイム検知アルゴリズムの提案
% \item 既存対策アプローチとDNSの潜在的データ転送脅威モデルの検討
%\end{itemize}
%\subsubsection{脅威モデル}
%\subsubsection{仮説}




\subsection{本論構成}
%\ref{kako}節では、過去における研究について述べ、
%\ref{kadai}章では、現状と今後の課題について述べる。
%また、付録\ref{omake1}におまけその1を添付する。
本稿の構成は以下の通りである.
まず第1章で,準備として,DNSプロトコル・秘匿通信・Tunnelingメカニズム・分散データベースの4点について説明する.
第2章では,先行研究が採用する検知アプローチにおけるLow Throughput手法・Slow Tunneling手法に対する課題を説明する.
第4章で提案手法とその実装について述べ,第5章で提案手法の性能評価と考察行い,第6章で残留する脅威モデルについて議論する.
最後に,第7章で結論と今後の課題について述べていく構成になっている.



\newpage
\section{準備}
本章は第3章以降の要素補足を目的に,本論において核となる技術内容・特徴およびそのメカニズムについて説明する.
\subsection{DNS}
\subsubsection{プロトコル概要}
DNSは,インターネットに接続された無数のコンピュータを一意に識別するためのIPアドレスを,人が認識しやすいドメイン名に変換する機能を担うネットワークプロトコルスタックの一つである.

ドメイン名は,ドット区切りで最大63文字のラベルで構成され,各ラベルは右から順にルートを頂点とする階層的な序列が表現されている.
ドメイン名の構造は,ルートを親ノードとして,その下に子ノードとしてTLD(Top Level Domain),さらにその下の子ノードとしてSLD(Second Level Domain)という具合に伸びるツリー構造である.
DNSでは,レコード情報を管理する主体を権威サーバと呼び,各権威サーバは下位のドメインに委任することで,
それぞれのノードは自身の下位に位置づくゾーンを管理することで,
すなわち,階層ごとに管理する主体
であり,下位のドメインに管理主体を委任することによって,委任された側は自身の下位のゾーンを管理する主体になる.
すなわち,
このように,管理主体は委任という仕組みによって管理するゾーンが分散・細分化されており,管理されるレコード情報は
DNSは分散データベースと言い換えることができる.
%現在IPアドレスには,32bitの名前空間を持つ10進数で構成されるIPv4(E.g. ``192.168.0.1")と,128bitの名前空間を持つ16進数で構成されるIPv6(E.g. ``2001:200:16a:8::230")がある.
%このような10・16進数で構成されるIPアドレスは,バイナリで解釈するコンピュータには都合がよい一方で,一般に人には記憶・認識することが困難である.
%ドメイン名は,このようなIPアドレスの特徴に対して,アルファベットを
%このようなIPアドレスのこのような特性に対して,より人が認識しやすいようにしたのがドメイン名である.
%DNSが担う役割は,IPアドレスをドメイン名に対応づける機能である.
%DNSは,各権威サーバがドメインで表現されるゾーンを管理することによる
具体的に,ドメイン名は,最大63文字で構成されるラベルをドット区切りで表現される(E.g. ``www.example.com").



DNSは,各権威サーバが固有のゾーンを管理することによる階層型分散データベースとして機能させている.
ドメインは通常,複数のラベルで構成されており,ラベルの区切り文字にはドットが使用されている.
階層の最上位に位置するルートはラベルを持たず,最も右に位置するラベルがTLDである.
最も一般的なレコードは,Aレコードであり,FQDNをIPv4アドレスにマッピングする.
ゾーンは,管理者が管轄するべき名前空間を意味する.
権威は,サブドメインへ委譲することが可能である.この機能は,NSレコードによって実現される.
\subsubsection{DNS Tunneling メカニズム}
\subsubsection{DNS Tunneling 特徴}
%\subsubsection{その他課題}
%\subsection{秘匿通信}
%秘匿通信(英Covert Channel)とは,
%情報転送を実現するにあたり,データの転送を本来の設計としていないプロトコルにそのデータを注入する手法である.
%\subsubsection{ステガノグラフィ}
%\subsubsection{代替プロトコル}
\subsection{分散ハッシュテーブル}
\subsubsection{アルゴリズム}
\subsubsection{暗号学的ハッシュ関数}
\subsection{P2P}
\subsubsection{アーキテクチャ}


\newpage
\section{先行研究}
\subsection{検知アプローチ}
\subsubsection{パターンマッチング}
\subsubsection{同一ドメインあたりのクエリ頻度}
\subsubsection{Qnameにおける文字列分布}
\subsubsection{Qnameにおける長さとエントロピー}
\subsubsection{Low ThroughputなTunnelingに対する検知手法}
\subsection{DNSアーキテクチャに基づく緩和アプローチ}
\subsubsection{Blockchainベース - Namecoin, Blockstack}
\subsubsection{P2Pベース - GNS(Gnu Name System)}



\newpage
\section{スーパーノード型P2Pに基づく名前解決システム}
%\subsection{分散データベースを用いた再起問い合わせにおけるハッシュ関数の適用}
%\subsection{本研究の位置づけ}
%\subsection{脅威モデル}
\subsection{設計}
\subsection{アーキテクチャ}
\subsection{プロトコルとメソッド}


\newpage
\section{評価}
\subsection{DNS Exfiltrationに対する定性評価}
\subsection{シミュレーション実験に基づく定量評価}
\subsubsection{シミュレーション実験構成}
\subsubsection{肥大化したリクエストペイロードサイズ}
\subsubsection{RTT(Round Trip Time)}
\subsubsection{トラフィック量}

\newpage
\section{議論}
\subsection{最適なハッシュ計算ノード}
\subsection{流入通信に対するリソースレコード}
%\subsubsection{CNAME, MX}
%\subsubsection{DNSKEY - 公開鍵検証}
%\subsubsection{TXT - ドメイン検証}

\newpage
\section{結論}
\subsection{まとめ}
\subsection{今後の課題}


% ===================謝辞===================
\newpage
\acknowledgements
ご指導ご鞭撻賜りありがとうございました.




% ===================参考文献===================
% ここでは \reference を使って、自分でリストを作るか、BibTeX を使って
% リストをつくって下さい。この例では BibTeX を作るような形式になってい
% ます。
\newpage
%\reference
%\bibliographystyle{plain}
\begin{thebibliography} {25}\small
 \bibitem{rfc1034} P.V. Mockapetris. ``Domain names - concepts and facilities. RFC 1034 (INTERNET STANDARD)," November 1987. Updated by RFCs 1101, 1183, 1348, 1876, 1982, 2065, 2181, 2308, 2535, 4033, 4034, 4035, 4343, 4035, 4592, 5936.
 \bibitem{rfc1035} P.V. Mockapetris. ``Domain names - implementation and specification. RFC 1035 (INTERNET STANDARD)," November 1987. Updated by RFCs 1101, 1183, 1348, 1876, 1982, 1995, 1996, 2065, 2136, 2181, 2137, 2308, 2535, 2673, 2845, 3425, 3658, 4033, 4034, 4035, 4343, 5936, 5966, 6604."
 \bibitem{covertchannel} ICANN, ``What Is an Internet Covert Channel?, " August 2016. https://www.icann.org/news/blog/what-is-an-internet-covert-channel
 \bibitem{rfc7626} S. Bortzmeyer. ``DNS Privacy Considerations, " August 2015.
 \bibitem{frameworkpos} KrebsonSecurity. ``Deconstructing the 2014 Sally Beauty Breach," May 2015. https://krebsonsecurity.com/2015/05/deconstructing-the-2014-sally-beauty-breach/. (accessd 2019-11-30).
 \bibitem{bondupdater} IronNet. ``Chirp of the PoisonFrog," February 2019. https://ironnet.com/blog/chirp-of-the-poisonfrog/. (accessd 2019-11-30).
 \bibitem{bernhardpos} Nick Hoffman. ``BernhardPOS," July 2015. https://securitykitten.github.io/2015/07/14/bernhardpos.html. (accessd 2019-11-30).
 \bibitem{multigrainpos} Fireeye. ``MULTIGRAIN – Point of Sale Attackers Make an Unhealthy Addition to the Pantry," April 2016. https://www.fieeye.com/blog/threat-research/2016/04/multigrain\_pointo.html. (accessd 2019-11-30).
 \bibitem{pisloader} Palo alto Networks. ``New Wekby Attacks Use DNS Requests As Command and Control Mechanism," May 2016. https://unit42.paloaltonetworks.com/unit42-new-wekby-attacks-use-dns-requests-as-command-and-control-mechanism/. (accessd 2019-11-30).
 \bibitem{denis} Kaspersky. ``Use of DNS Tunneling for C\&C Communications," April 2017. https://securelist.com/use-of-dns-tunneling-for-cc-communications/78203/. (accessd 2019-11-30).
 \bibitem{dnsmessenger} CISCO Talos. ``Spoofed SEC Emails Distribute Evolved DNSMessenger," October 2017. https://blog.talosintelligence.com/2017/10/dnsmessenger-sec-campaign.html. (accessd 2019-11-30).
 \bibitem{udpos} Cylance. ``Threat Spotlight: Inside UDPoS Malware," Febrary 27 2018. https://threatvector.cylance.com/en\_us/home/threat-spotlight-inside-udpos-malware.html. (accessd 2019-11-30).
 \bibitem{born} K. Born and D. Gustafson, ``NgViz: detecting DNS tunnels through n-gram visualization and quantitative analysis," Proceedings of the Sixth Annual Workshop on Cyber Security and Information Intelligence Research, Oak Ridge, Tennessee, 2010, pp. 1-4.
 \bibitem{cheng} Cheng Qi, Xiaojun Chen, Cui Xu, Jinqiao Shi, Peipeng Liu, ``A Bigram based Real Time DNS Tunnel Detection Approach," Procedia Computer Science, Volume 17, 2013, Pages 852-860.
 \bibitem{liu} J. Liu, S. Li, Y. Zhang, J. Xiao, P. Chang and C. Peng, ``Detecting DNS Tunnel through Binary-Classification Based on Behavior Features," 2017 IEEE Trustcom/BigDataSE/ICESS, Sydney, NSW, 2017, pp. 339-346.
 \bibitem{asaf} Asaf Nadler, Avi Aminov, Asaf Shabtai, ``Detection of malicious and low throughput data exfiltration over the DNS protocol," Computers \& Security, Volume 80, 2019, Pages 36-53.
 \bibitem{steadman} J. Steadman and S. Scott-Hayward, ``DNSxD: Detecting Data Exfiltration Over DNS," 2018 IEEE Conference on Network Function Virtualization and Software Defined Networks (NFV-SDN), Verona, Italy, 2018, pp. 1-6.
 \bibitem{jawad} J. Ahmed, H. H. Gharakheili, Q. Raza, C. Russell and V. Sivaraman, ``Monitoring Enterprise DNS Queries for Detecting Data Exfiltration from Internal Hosts," in IEEE Transactions on Network and Service Management.
 \bibitem{ozymandns} ``OzymanDNS - Tunneling SSH over DNS," https://room362.com/post/2009/2009310ozymandns-tunneling-ssh-over-dns-html/, accessed: 2019-11-20.
 \bibitem{iodine} ``iodine," http://code.kryo.se/iodine/, accessd: 2019-11-20.
 \bibitem{dnscat2} ``DNScat2," https://github.com/iagox86/dnscat2, accessed: 2019-11-20.
\end{thebibliography}
\bibliography{mthesis}




% ===================付録===================
\appendix

\section{発表リスト(国内研究会)}
%\subsection{国内研究会}
\begin{enumerate}
 \item \underline{高須賀 昌烈}, 妙中 雄三, 門林 雄基, ``非実在ドメインに対するネガティブキャッシュの拡張と再帰問い合わせハッシュ化の提案", 電子情報通信学会 情報ネットワーク研究会, 2019-10-ICTSSL-IN, 2019年10月.
\end{enumerate}
%\label{omake1}
%これはおまけです。これはおまけです。これはおまけです。これはおまけです。
%\begin{figure}
%\centerline{これはおまけの図です。}
%\caption{おまけの図}
%\end{figure}
%\section{おまけその2}
%これもおまけです。これもおまけです。これもおまけです。これもおまけです。



\end{document}

