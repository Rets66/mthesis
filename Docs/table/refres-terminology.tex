\begin{table}[h]
 \caption{DNS-TDにおける用語}
 \centering
  \begin{tabular}{ll}
    \toprule
    \multicolumn{1}{c}{\textbf{表記}} & \multicolumn{1}{c}{\textbf{意味または機能}}\\
    \midrule

    コンテンツ & \begin{tabular}{l}・識別子に関連づけられたレコード情報の実体\end{tabular}\\ \hline

    コンテンツID & \begin{tabular}{l}・識別子\end{tabular}\\ \hline

    ドメインID & \begin{tabular}{l}・識別子(コンテンツIDが重複した際に使用)\end{tabular}\\ \hline

    レコード情報 &
      \begin{tabular}{l}
        ・リソースレコードの具体的な値\\
        $\,$ (例 IPアドレス)
      \end{tabular}\\ \hline

     リソースレコードタイプ &
      \begin{tabular}{l}
        ・オブジェクトに関連づけるリソースレコードの型\\
        $\,$ (例 A, AAAA, MX)
      \end{tabular}\\ \hline

    オブジェクト &
      \begin{tabular}{l}
       ・問い合わせる対象\\
       $\,$ (ドメイン名もしくはIPアドレス)
      \end{tabular}\\ \hline

    スタブリゾルバ & \begin{tabular}{l}・名前解決クライアント\end{tabular}\\ \hline

    フルサービスリゾルバ &
      \begin{tabular}{l}
       ・スタブリゾルバからのクエリハンドリング\\
       ・識別子の作成
      \end{tabular}\\ \hline

    マネージャ &
      \begin{tabular}{l}
       ・フルサービスリゾルバからのクエリハンドリング\\
       ・ゾーンの管理\\
       ・コンテンツの保持
      \end{tabular}\\ \hline

    プロバイダ & \begin{tabular}{l}・コンテンツの作成・更新・削除操作\end{tabular}\\

    \bottomrule
  \end{tabular}
 \label{tab:refres-terminology}
\end{table}
