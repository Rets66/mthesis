\begin{table}[htb]
 \caption[マネージャとゾーンの対応表]{6つのマネージャによって管理されるハッシュテーブルにおいて,マネージャの情報とそのマネージャが管理するゾーンが記載された対応表の例}
 \centering
  \begin{tabular}{lrl}
    \toprule
		\multicolumn{1}{c}{\textbf{ゾーン}} & \begin{tabular}{c}\textbf{マネージャ}\\\textbf{アドレス}\end{tabular} & \multicolumn{1}{c}{\textbf{ドメイン}} \\
    \midrule
    (000…00, 2zz…zz) & 192.35.51.30 & com.  \\
		\multicolumn{1}{c}{...} & \multicolumn{1}{c}{...} & ... \\
    (500…00, 6zz…zz) & 192.5.6.30 & net. \\
		\multicolumn{1}{c}{...} & \multicolumn{1}{c}{...} & ... \\
    (b00…00, czz…zz) & 199.249.112.1 & org. \\
		\multicolumn{1}{c}{...} & \multicolumn{1}{c}{...} & ... \\
    (n00…00, mzz…zz) & 199.254.31.1 & info. \\
		\multicolumn{1}{c}{...} & \multicolumn{1}{c}{...} & ... \\
    (y00…00, zzz…zz) & 194.0.0.53 & arpa. \\
    \bottomrule
  \end{tabular}
 \label{tab:hash-management}
\end{table}
