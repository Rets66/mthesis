\section{提案手法}
%第\ref{sec:related-works}で述べたように,これまでに提案されてきた検知に基づくDNS Tunneling対策には,Low Throughput手法および転送頻度を下げる手法に対して,検知が困難であるという課題がある.
%他方,新しいアーキテクチャに基づく名前解決システムには,マイグレーションの課題が残留している.

本章では,DNS Exfiltration(\ref{sec:dns-exfiltration})の発生抑制を目的として,分散ハッシュテーブルを利用した名前解決システムD-NRS(Destributed hash table based Name Resolution System)について説明する.

\subsection{アーキテクチャ}
D-NRSは,サーバ同士が相互連携するサーバ群とクライアントで構成されるクライアントサーバアーキテクチャで動作する.
D-NRSにおいて,スタブリゾルバとフルサービスリゾルバとの間の手続きは,既存のDNSと変わらない.
しかし,D-NRSのサーバは,フルメッシュのネットワーク基盤のもとで各サーバが相互に連携することで,リソース情報を分散して管理する.
%既存のDNSにおける名前解決の仕組みでは,階層的なドメイン構造に基づきルートから順に権威サーバに対し再帰的に問い合わせていくことで,スタブリゾルバからの問い合わせレコード情報を保持するサーバのアドレスを特定する.
D-NRSにおけるレコード情報は,QNAMEとレコードタイプを引数として,ハッシュ関数によって算出されるコンテンツIDがシンボルとして紐づけられる.

全てのフルサービスリゾルバは,シンボルとそのシンボルに紐づいたレコード情報を管理するサーバを対応づけた表を保持し,この対応表に基づきコンテンツIDからサーバを一意に特定する.
複数のサーバが,アルファベットおよび数字の順序(a $\rightarrow$ z, 0 $\rightarrow$ 9)で並んだハッシュテーブルの連続した範囲を管理し合い,担当の範囲下にあるコンテンツIDに紐づいたレコード情報を担当サーバが管理する.
対応表\ref{tab:hash-management}には,どこからのどこまでのハッシュテーブルの範囲をどのサーバが管理するのかについて記述されている.

\begin{table}[htb]
 \caption[マネージャとゾーンの対応表]{マネージャの情報とそのマネージャが管理するゾーンが記載された対応表の例}
 \centering
  \begin{tabular}{lrl}
    \toprule
		\multicolumn{1}{c}{\textbf{ゾーン}} & \begin{tabular}{c}\textbf{マネージャ}\\\textbf{アドレス}\end{tabular} & \multicolumn{1}{c}{\textbf{ドメイン}} \\
    \midrule
    (000…00, 2zz…zz) & 192.35.51.30 & com.  \\
		\multicolumn{1}{c}{...} & \multicolumn{1}{c}{...} & ... \\
    (500…00, 6zz…zz) & 192.5.6.30 & net. \\
		\multicolumn{1}{c}{...} & \multicolumn{1}{c}{...} & ... \\
    (b00…00, czz…zz) & 199.249.112.1 & org. \\
		\multicolumn{1}{c}{...} & \multicolumn{1}{c}{...} & ... \\
    (n00…00, mzz…zz) & 199.254.31.1 & info. \\
		\multicolumn{1}{c}{...} & \multicolumn{1}{c}{...} & ... \\
    (y00…00, zzz…zz) & 194.0.0.53 & arpa. \\
    \bottomrule
  \end{tabular}
 \label{tab:hash-management}
\end{table}


フルサービスリゾルバはQNAMEとリソースレコードタイプをキーとしてサーバに問い合わせるバリューが応答されるKVSモデルに基づく名前解決システムである.

QNAMEとリソースレコードタイプに基づき生成されるハッシュ値をキー,分散ハッシュテーブル上で保存されているレコード情報をバリューとするKVSモデルに基づく名前解決システムである.

D-NRSでは,従来のDNSのエコシステムの内,スタブリゾルバからリカーシブサーバまでの手続きを継承することで,エッジノードにおけるシステムのマイグレーションに伴う課題を軽減する.
全てのレコード情報は,レコードタイプとドメイン名もしくはIPアドレスの文字列和(rtype+domain, rtype+ipaddress)をハッシュ関数に適用して算出されるハッシュ値をコンテンツIDとして,レコード情報に紐づっけr.




スタブリゾルバは既存のDNSと変更はなく,DNSクライアントとして,名前解決を依頼する主体として位置づくノードである.
リカーシブサーバは,スタブリゾルバからの問い合わせに対してリソース情報を保持する主体に代理的に問い合わせ機能と,問い合わせた情報を一定期間キャッシュするキャッシュサーバとして機能するノードである.
既存のDNSにおける権威サーバは,リカーシブサーバからの問い合わせに応答するマネージャと,リソース情報について作成・消去および更新などの操作をするプロバイダの二つに分けられる.
D-NRSにおいて,リソース情報は,オブジェクト(object)とリソースレコードタイプ(rtype)を引数とするハッシュ関数から算出されるコンテンツIDが紐づけられ,そのコンテンツIDに基づきハッシュ空間上に対応づけられる.
各マネージャは,ハッシュテーブル全体のうち連続した幾らかの管理範囲が割り当てられ,範囲下にあるコンテンツIDに基づいたリソース情報を保存・管理する.
このようにして,リソース情報は,特定の範囲ごとに分割されたハッシュテーブルにて分散的に管理される.
マネージャ同士は,フルメッシュなネットワーク構造で接続し合い,各マネージャには地理的・意味的に類似なプロバイダが階層的な序列に基づき接続される.

プロバイダからリソース情報への操作リクエストがあった際には,リソース情報のコンテンツIDを算出し,そのIDが含まれるハッシュ空間を管理する担当マネージャに操作依頼を転送し,受け取った担当マネージャは直ちに,リソース情報への操作を実行する.


%\begin{figure}[h]
% \centering
% \includegraphics[width=10.0cm]{images/spnrs-architecture.png}
% \caption{D-NRSによって,名前解決している様子.}
% \label{fig:spnrs-architecture}
%\end{figure}


%\subsubsection{想定する脅威}
%\subsubsection{名前解決メカニズム}
\subsection{プロトコル}
\subsection{分散ハッシュテーブル}
\subsection{ハッシュ範囲に対する管理ノードの対応表}
%\subsubsection{QNAMEとRRを引数とするハッシュ値をキーとするクエリ}
%\subsection{データベース}
