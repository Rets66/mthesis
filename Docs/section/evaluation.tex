\section{評価}
\subsection{実装}
% 
\subsection{実験環境}
\subsection{DNS Tunnelingに対する理論評価}
\subsubsection{評価要素 : 対Exfiltration}
\subsubsection{評価要素 : 対Infiltration}
\subsection{DNSとの比較に基づく特性評価}
\subsubsection{評価要素 : RTT(Round Trip Time)}
従来のDNSでは,ラベルごとにゾーンが移譲されている場合,レコード情報を保持するノードまでのホップ数はラベル数nに比例する.
それに対して,SORESでは,識別子から一意にレコード情報を保持するマネージャノードを特定できるので,常にホップ数は2である.
このため,既存の名前解決システムより速度の向上が期待される.
% 実装の必要はない
% 既存のDNSにおけるラウンドトリップのうち,再帰問い合わせの最後の権威サーバRTTがSORESのフルリゾルバとマネージャのそれになる.
% 遠いものと近いもののRTTを用意する必要がある
\subsubsection{評価要素 : パケットサイズ}
% 実装する必要はない.理論で評価できる
% あえてするとすれば,フルサービスリゾルバとマネージャ間通信
% DDoSへの影響については,リクエストとレスポンスパケットのサイズからアンプ率に着目する
% アンプ率は,既存のDNSと基本的に変わらない.DNSSECがあるかないかについて議論するくらい
SORESでは,224bitを固定長とするコンテンツIDをシンボルとすることによって,レコード情報にアクセスする.
この仕組みの影響で,SORESのパケットは従来のパケットと比較して肥大する特性がある.
このため,送信元を目的ホストと偽装することで目的ホストの計算リソースを圧迫するDDoS攻撃に対して,脅威を高める可能性が予想される.
\subsubsection{評価要素 : トラフィック量}
% SORESについては,実装の必要性はない
% ただし,DNSについてトラフィックデータを収集する必要がある
% 必要なデータ
% * 一定期間のSORESトラフィック
% * 同一期間のDNSトラフィック
% 比較項目
% * 平均したドメイン長と
SORESでは,シンボル志向の名前解決メソッドによって,既存の再帰問い合わせによるメソッドよりも少ないトラフィックに抑えることが期待される.
SORESでは,クエリ数とトラフィック量は比例関係にある.
他方で,新たにマネージャ間通信という従来にはないトラフィックが発生する.
本項では,これトラフィックがネットワーク全体にどの程度影響を及ぼすのかについて評価する.
% マネージャが攻撃された場合の影響は,世界規模に影響する.
% 従来のドメインごとにゾーンが分離している設計と違い,一つのゾーンには様々な組織のドメインが管理されている.
% このゾーンを全て書き換えることができることは脅威である.
