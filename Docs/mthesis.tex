\documentclass[12pt]{jarticle} % Japanese
%\documentclass[12pt]{article} % English
% if there are problems in the above regarding fonts, use this
% \documentclass[UTF8]{ctexart}


%package
\usepackage{amssymb, amsmath, newtxmath}
\usepackage[ruled, vlined]{algorithm2e}
%\usepackage{algorithm}
\usepackage{cite}
\usepackage[utf8]{inputenc}
%\usepackage{utf}
\usepackage{naist-jmthesis} %Japanese
%\usepackage{naist-mthesis} %English
\usepackage[dvipdfmx]{graphicx}
\usepackage[dvipdfmx]{hyperref}
\usepackage{pxjahyper} %Required pxjahyper.sty
\usepackage[dvipdfmx]{xcolor}
\usepackage{pxjahyper} %TOC文字化け対策
\usepackage{booktabs}
\usepackage{lscape}
\usepackage{here}
\usepackage{colortbl}


\SetAlgorithmName{アルゴリズム}{アルゴリズム}{アルゴリズム目次}
\DontPrintSemicolon
\RestyleAlgo{algoruled}
\SetAlgoCaptionSeparator{.}
\SetFuncSty{textbf}
\SetKwIF{If}{ElseIf}{Else}{if}{:}{elif}{else:}{}
\SetKwFor{For}{for}{\string:}{}
\SetKwFor{While}{while}{\string:}{}
\SetKw{To}{to}
\SetKw{Break}{break}
\SetKw{Continue}{continue}
\SetKw{True}{True}
\SetKw{False}{False}

%definition
\definecolor{purple}{RGB}{98, 114, 164}
\hypersetup{
 colorlinks=true,
 linkcolor=black,
 citecolor=purple,
 urlcolor=black,
 filecolor=purple,
 anchorcolor=purple,
 pdfborder={0, 0, 1},
 linktoc=all
}
\renewcommand\thefootnote{\textcolor{purple}{*\arabic{footnote}}}


% Page style
\pagestyle{final}       % Camera-Ready
%\pagestyle{draft}      % Draft
\lang{Japanese} % Japanese
%\lang{English} % English
% Student Number
\studentnumber{1811147}
\doctitle{\mastersthesis}% 修士論文
\major{\engineering}     % 工学



% 日本語題目 (in LaTeX)
\title{トンネリング抑止を目的とした\\分散ハッシュテーブルを利用したDNSに関する研究}
%\title{クエリ透過性に着目したDNSトンネリング抑止に関する研究}
% 日本語題目 (in plain text)
%   注: (in LaTeX)と同じ場合は指定する必要なし。
%       この情報は修士論文/課題研究には現れませんが、管理のために必要です。



% 英語題目 (in LaTeX)
%\etitle{Proposal for Mitigation of DNS Exfiltraion using Hash Function to Recursive Name Resolution}
\etitle{A Study of Domain Name System \\using Distributed Hash Table \\for Tunneling Deterrence}
% 英語題目 (in plain text)
%   注: (in LaTeX)と同じ場合は指定する必要なし。
%       この情報は修士論文/課題研究には現れませんが、管理のために必要です。
%\eptitle{Theoretical Studies on Low-Speed Calculation Algorithms of pi \\
%Utilizing the Sun and the Moon}
%\eptitle{Proposal for Mitigation of DNS Exfiltraion using Hash Function to Recursive Name Resolution}



% 日本語氏名 (in LaTeX)
%   (姓と名の間に空白を入れて下さい)
\author{高須賀 昌烈}
%\pauthor{}
%   (first name, last name の順に記入し、先頭文字のみを大文字にする。)
\eauthor{Shoretsu Takasuka}
% 別の例: \eauthor{Kurt G\"{o}del}
%\epauthor{}



% 論文提出年月日
\syear{2020}
\smonth{3}
\sday{12}



% 専攻の選択
%\department{\infproc}  % 情報処理学
%\department{\infsys}    % 情報システム学
%\department{\bioinf}   % 情報生命科学
\department{\infsci}    % 情報科学



% 審査委員(日本語)
%   (姓と名、名と称号の間に空白を入れて下さい)
%5人以上の場合,5人目以降は\addcmembers を使って宣言する。
%最大で合わせて8人まで宣言可能。
%主指導教員、副指導教員を明記する。両指導教員以外は委員。
%学外審査委員は、大学名を明記する
% 4人の場合
\cmembers{門林 雄基 教授}{(主指導教員)}
         {笠原 正治 教授}{(副指導教員)}
         {林 優一 教授}{(副指導教員)}
         {妙中 雄三 准教授}{(副指導教員)}
% 3人の場合
%\cmembers{○○ ○○ 教授}{(主指導教員)}
%         {○○ ○○ 教授}{(副指導教員)}
%         {○○ ○○ 准教授}{(副指導教員)}
%         {}{}



% 審査委員(英語)
%     (first name, last name の順に記入し、先頭文字のみを大文字にする。
%       first name と last name の間に空白、
%       last name と 称号の間にカンマと空白を入れて下さい。)
% 5人以上の場合,5人目以降は\eaddcmembers を使って宣言する
% Supervisor, Co-supervisor, and Member must be specified.
% 4人の場合
\ecmembers{Professor Youki Kadobayashi}{(Supervisor)}
          {Professor Shoji Kasahara}{(Co-supervisor)}
          {Professor Yu-ichi Hayashi}{(Co-supervisor)}
          {Associate Professor Yuzo Taenaka}{(Co-supervisor)}
% 3人の場合
%\ecmembers{Professor xx xx}{(Supervisor)}
%          {Professor xx xx}{(Co-supervisor)}
%          {Associate Professor xx xx}{(Co-supervisor)}
%          {}{}



% ===================キーワード===================
\keywords{ドメインネームシステム,DNSトンネリング,ネットワークセキュリティ.分散ハッシュテーブル}
% キーワード5〜6個 (in plain text)
%   注: (in LaTeX)と同じ場合は記入する必要なし。
%       この情報は修士論文/課題研究には現れませんが、管理のために必要です。
%\pkeywords{pi, 天文学, 数学, 計算機, アルゴリズム}
%\pkeywords{DNS Exfiltration, 秘匿通信,ハッシュ関数,再帰問い合わせ}
% 5 or 6 Keywords (in LaTeX)
%\ekeywords{$\pi$, astronomy, mathematics, computer, algorithm}



% ===================Keyword===================
\ekeywords{Domain Name System(DNS), DNS Tunneling, Network Security, Distributed Hash Table}
% 5 or 6 Keywords (in plain text)
%   注: (in LaTeX)と同じ場合は記入する必要なし。
%       この情報は修士論文/課題研究には現れませんが、管理のために必要です。
%\epkeywords{pi, astronomy, mathematics, computer, algorithm}
%\epkeywords{DNS Exfiltration, Covert Channel, Hash Function, Recursive Name Resolution}



% ===================内容梗概===================
\abstract{
 巧妙化するサイバー攻撃の手法の中に,攻撃通信を無害な通信に偽装することで検知を迂回する手法がある.
 DNSトンネリングと呼ばれる手法は,そのような秘匿通信手法の中で最も広く利用されている.
 %DNSトンネリングは,DNSクエリパケットのドメイン名あるいは応答パケットのリソースレコードにデータを含めることによって,データを転送するという単純な仕組みで動作する.
 この手法に対して,従来は,サブドメインの長さおよびエントロピー,トラフィック頻度などを特徴量とする閾値や悪性モデルを推定する検知アプローチが取られてきた.
 しかし,転送データ量の削減やパケット間のインターバルを長期化させるといった転送効率を下げる手法を用いることで,既存の検知アプローチを迂回される脅威がある.

 この脅威に対して,本研究ではDNSトンネリングの発生を抑止する名前解決システムを提案する.
 システムに採用したアプローチは,権威サーバの機能を分離させることでスタブリゾルバからサーバへのクエリ透過を抑制し,このメカニズムによってDNSトンネリングの発生を抑止することができる.
 %し,ドメイン名とレコードタイプの組を表す識別子をフラットな名前空間上にハッシュ関数によって写像することによって,トンネリング抑止の機能を提供する.
 評価では,実装した提案システムのプロトタイプ上でトンネリング通信をシミュレーションさせることによって,DNSトンネリングの通信抑止に有効であることを示した.
 また,既存システムとの比較に基づいた特性の評価を行った結果から,提案システムにはトラフィック量の削減と高速な名前解決という優位性があることを示した.
 %提案システムは,ルート権威サーバに通信が集約する既存システムにおける課題を解消し,既存システムと比べて高速な名前解決として利用されることが期待される.
}
%   注: 行の先頭が\\で始まらないようにすること。
%   注: (in LaTeX)と同じ場合は記入する必要なし。
%       この情報は修士論文/課題研究には現れませんが、管理のために必要です。
%       改行する箇所には空白行を入れる。
%       行の先頭が\\で始まらないようにすること。
%\pabstract{
%}
% Abstract (in LaTeX)
%  注:  行の先頭が\\で始まらないようにすること。



% ===================Abstruct===================
\eabstract{
 In well-organised cyber-attacks, camouflage techniques are used to bypassing the monitoring or analysis systems.
 DNS Tunneling is the most popular method in those covert network techniques.
 It's utilised a simple concept which transfers data through the Qname field in the DNS query packet.
 Many solutions have been proposed to counter against this 	tunneling technique.
 However, those previous research used some tactics such as subdomain's length or entropy, and frequency of traffic by analysing tunneling traffic.
 The previous countermeasure were based on the detection apploach estimated threshold or malicious models.
 However those detection apploachies are exposed to the menace of the being bypassed.
 For example, the malicious user can bypass by reducing the transfer data in a packet or leaving times between query packets and response packets.

 In this study, I propose name resolution system based on distributed hash table for DNS tunneling deterrence.
 The core idea is the division of authorization server function into contents management and records cofiguration.
 Following this approach, the transparency from clients to servers which malicious user configured is restricted, tunneling queries don't reach the server in the end.
 I implemented prototype system of the proposed design for the evaluation, and verified the effectiveness for the tunneling deterrence by simulation tunneling test on the prototype.
 Besides the results, I evaluated the features of the proposed system shows my proposed system is expected for a reduction of name resolution traffic and faster name resolution.
}
% Abstract (in plain text)
%   注: (in LaTeX)と同じ場合は記入する必要なし。
%       この情報は修士論文/課題研究には現れませんが、管理のために必要です。
%       改行する箇所には空白行を入れる。
%       行の先頭が\\で始まらないようにすること。
%\epabstract{
%The calculation of pi has been paid much attention since human beings
%appeared on the earth.
%This thesis presents novel low-speed algorithms to calculate
%pi utilizing the sun and the moon.
%This is a sample abstract. This is a sample abstract.
%}



% ===================表紙===================
\begin{document}
\titlepage
\cmemberspage
\firstabstract
\secondabstract



% ===================目次===================
\toc
\newpage
\listoffigures
\newpage
\listoftables
\listofalgorithms



% ===================本文===================
\newpage
\pagenumbering{arabic}

\section{背景}
増加し続けるサイバー攻撃に対して,現在多くの組織は,SIEMのようなネットワークトラフィックを監視するシステムから発せられるアラートを処理することでセキュリティの脅威に対処している.
一方,機密情報の奪取や諜報活動を行う攻撃者は,そのような監視システムを迂回するために秘匿通信手法を用いることが知られている~\cite{mitre-custom-c2}.
このような攻撃に対して,検知の閾値や悪性モデルを調整するアプローチが考えられるが,誤検知とのトレードオフの関係にある.

秘匿通信の代表的な手法にDNSトンネリングがある.
DNS(Domain Name System)は,IPアドレスをはじめとしたドメイン名に関連づけられたリソースレコードを解決するシステムである.
この名前解決の機能によって,ユーザは識別しづらいIPアドレスを直接利用することなくサーバのリソースにアクセスすることができる.
インターネットの利活用において,名前解決の通信はメールの送受信やウェブ検索などの通信に先立って行われる.
すなわち,DNSのトラフィックをフィルタリングすることはインターネットの利活用に大きな影響を及ぼすため,容易にフィルタリングを行うことが困難であるという特性がある.
DNSトンネリングは,フィルタリングされにくいというDNSのこの特性を利用する.
DNSトンネリングでは,クライアントからサーバ方向のデータ転送にQnameを用い,その逆方向のデータ転送にはリソースレコードを使用する.
このように双方向にデータを転送できるDNSトンネリングは,ターゲット組織から取得したデータを外部に流出させる際の手段としてだけでなく,ターゲットネットワークに潜伏しているマルウェアに対するC2通信の手段として,サイバー攻撃で広く利用されている.

しかしDNSプロトコルは,その名前解決の仕組みは変更されずに現在まで使用され続けている.
このDNSトンネリングに対して,同一ドメインへの時間あたりのトラフィック頻度や問い合わせられるドメイン名のサブドメインにおける文字列の分布や長さといった特徴に基づく検知アプローチはこれまでに多数提案されている~\cite{born, cheng, liu, asaf, steadman, jawad}.
検知に基づくアプローチを取る先行研究では,DNSトンネリングの擬似通信として,Githubなどから入手可能なトンネリング実装プログラムが用いた評価が行われる.
しかし,擬似通信の発生に広く使われるIodine~\cite{iodine}やDNSCat2~\cite{dnscat2}といった実装は,インタラクティブシェル機能を目的としているため,時間あたりのトラフィック量が多く,パケットサイズも大きいという性質がある.
このような顕著な性質が現れるトンネリング実装に対して,パケットごとのインターバルを1ヶ月間などにトラフィック頻度を調整したり,正規のFQDNの平均の長さまで注入するデータ量を下げるなどのバイパス手法がある~\cite{asaf}.
これらバイパス手法を利用した場合,トンネリング実装の特徴に基づいた既存の検知手法を用いて検知することは困難である.

一方,これまでに多数の次世代名前解決システムは提案されてきているが,DNSトンネリング抑止を目的としたシステムは筆者が知りうる限り提案されていない.
そこで本研究では,秘匿通信手法であるDNSトンネリングの発生を抑止する次世代の名前解決システムを提案する.


\subsection{目的}
本研究の目的は,従来の名前解決システムを保ちながら,秘匿通信手法のDNSトンネリングを抑止する名前解決システムを開発することである.
既存のDNSによる名前解決システムは,現在のインターネットの根幹技術であるため,移行を念頭に設計されている必要がある.
%その名前解決エコシステムに大幅な変更を加えることは,高い導入コストが要求されるため望まれない.
例えば,期待されないシステムとしては,以下のようなものが予想される.
\begin{itemize}
 \setlength{\itemsep}{-0.5mm}
 \item DNSトンネリング抑止は実現されるが,既存の名前解決システムが実現されない
 \item DNSトンネリング抑止は実現されるが,既存システムからの大幅な変更が必要になり,未対応のコンピュータのインターネット接続に支障をきたす
\end{itemize}
以上のことを踏まえた既存システムとの互換性を確保しながら,目的の実現を目指す.

\subsection{貢献}
提案システムは,既存システムのクライアントサーバアーキテクチャに基づき,既存の再帰問い合わせの仕組みのみに改変処理を施すことで,DNSトンネリングを抑止する名前解決システムの実現させる.
提案システムは,既存クライアントに変更を加えずに秘匿通信としての機能を抑止することができるため,セキュアな名前解決システムとして広く一般に利用されることが期待される.


\section{脅威モデル}
\label{sec:dns-tunneling}
本章では,はじめにDNSの概要について述べる.
次に,本研究における脅威モデルであるDNSトンネリングについて説明する.

\subsection{DNSの概要}
\label{sec:dns-protocol}

DNSは,ドメイン名に関連づけられたリソースレコードを解決するシステムである~\cite{rfc1034, rfc1035}.
DNSがユーザから問い合わせられたドメイン名のIPアドレスを解決することで,ユーザは識別しづらいIPアドレス(IPv4:``93.184.216.34", IPv6:``2606:2800:220:1:248:1893:25c8:1946")を直接入力することなく,サーバにアクセスすることができる.
このような利便性を実現するDNSによる名前解決の機能は,ユーザがインターネットを利活用する上で極めて重要である.

\subsubsection{名前空間}
DNSにおいて,各種リソースレコードが関連づけられるドメイン名は,ルートを頂点とする逆ツリー構造で構成されている.
ドメインの序列を表記する際には,上位ドメインを右に,下位ドメインを左にする.


``example.com."を例にとると,ドットで表記されるルートが最も右に位置し,ルートの一つ下層に位置するTLD(Top Level Domain)である``com"が後続する.
ドメインの区切りには,ドットが使用され,TLDの一つ下層に位置づくSLD(Second Level Domain)として``example"が連結していることを``example.com."は表している.
図~\ref{fig:dns-architecture}に示す通り,全てのドメインには,ゾーンと呼ばれる名前空間があり,このゾーンはそれぞれのドメインに配置される権威サーバによって管理される.
DNSには,委譲という仕組みを利用することでゾーンを分割できる特徴がある.
例えば,``example"と``google"というドメインが連結している``com."について考える.
委譲の仕組みを使用しなかった場合には,``com."が``www.example.com."や``mail.google.com."といった``example.com."と``google.com."に連結した全てのドメイン名を管理する.
しかし,``com."がそれぞれのドメインにゾーンを委譲した場合,先の``www.example.com"は``example.com."が管理し,``mail.google.com."は``google.com."が管理することになる.
DNSでは,この委譲の仕組みを利用することで,ゾーンの肥大を抑止させ,サーバの負荷分散を実現する設計になっている.

\begin{figure}[th]
 \centering
 \includegraphics[width=12.0cm]{figure/dns-architecture.png}
 \caption[ゾーンごとに分割された名前空間]{ゾーンごとに名前空間を色で区分した様子.Root権威が管理するオレンジで表現されたゾーンには,comやorgをはじめとしたドメインが含まれる.Rootの直下に位置するTLDの一つであるcomドメインは,青色で囲まれたexampleドメインを含むゾーンを管理する.comに紐づけられるexampleドメインは,wwwなどを管理する.}
 \label{fig:dns-architecture}
\end{figure}

\textbf{アーキテクチャ}\\
DNSは,クライアント・サーバアーキテクチャで構成され,機能に基づき3つのサービスに分類することができる.
\begin{itemize}
 \item スタブリゾルバ
 \vspace{-3mm}
 \item フルサービスリゾルバ
 \vspace{-3mm}
 \item 権威サーバ
\end{itemize}

スタブリゾルバは,名前解決の問い合わせを行うクライアントノードである.
フルサービスリゾルバ(キャッシュサーバ・リカーシブサーバとも呼称される)は,スタブリゾルバに代わって,リソースレコードを保持する権威サーバに問い合わせるクライアントノードである.
名前解決をする際には,ルートから順にTLD,SLDという具合に権威サーバに再帰的に問い合わせることで,最終的に目的のドメイン名に関するリソースレコード情報を取得する.
この時,はじめのルート権威サーバのアドレスは``root.hints"と呼ばれるファイルに基づいて問い合わせるが,それより下位のドメインについては,上位の権威サーバが次の権威サーバのアドレスを応答することで名前解決のチェーンを繋げている.
すなわち,ルート権威サーバがTLDの権威サーバのアドレスを応答し,TLDの権威サーバがSLDの権威サーバーのアドレスを応答していく具合である.
権威サーバは,リソースレコードを保持するサーバノードであり,フルサービスリゾルバからの問い合わせ依頼に応答する.

\subsubsection{ドメイン名とリソースレコード}
%TLDごとに登録するプロセスや必要書類,金額は異なり,上位
ルートからホストまでの階層構造の繋がりは,``label3.label2.label1."のように右から左方向に連結することで表現される.
この表記は,FQDN(Fully Qualified Domain Name)と呼ばれ,右端のドットがルート,``label1"がTLD,``label2"がSLDを表現し,ドメインごとの階層にはドットが使用される.
ドメインの階層構造において,全てはルートを頂点としているため,ルートを意味するドットを略記した``label3.label2.label1"が,一般にはドメイン名として解釈される.
次にドメイン名に関する長さおよび使用できる文字列について説明する.
\vspace{-1pt}
\begin{eqnarray}
 (LabelD).(LabelC).(LabelB).(LabelA). \label{eq:domain-name} \\
 (Length) + (LabelD) + ... + (length) + (LabelB) + (length) + (LabelA) + 0 \label{eq:label-name} \\ 
 1 + (Max 63) + ... + 1 + (Max 63) + 1 + (Max 63) + 1 = (Max 255) \label{eq:length-label-domain}
\end{eqnarray}

(\ref{eq:domain-name})は,複数のラベルで構成されたドメイン名の例である.
(\ref{eq:label-name})は,Questionヘッダーに注入される際のそのドメイン名を表すデータである.
Questionセクションでは,ドメイン名を表す際にドットは省略され,ラベルの長さとラベル名,そしてドメイン名の終わりを意味する``0"で表現される.
(\ref{eq:length-label-domain})は,ラベルの長さとラベル名のサイズを表す.
ラベルの長さは,1バイトのサイズで表現され,ラベル自体の最大長は63バイトである.
Questionセクションの最大長255バイトは,ラベルの長さとラベル,そしてドメイン終了を表す``0"を含めた長さである.
このため,最初のラベル長を表す1バイトとドメイン名の終了を意味する``0"を表すための1バイトを差し引いた253バイトが,実際のドメイン長の最大長である.

ラベルには,数字とアルファベットおよびハイフン(``-")を使用することができ,ラベル中に大文字・小文字の区別はない.
他方で,アルファベットなどのASCII以外にも,国際化ドメイン名(IDN: Internationalized Domain Name)を使用すると日本語やアラビア語なども使用することができる.
IDNは,Punnycode\footnote{Punnycode: Unicode文字列を一意かつ可逆的にASCII文字列に変換する符号化方式}などのエンコーディング手法に基づき,DNSクエリする際にはASCIIコードに変換される~\cite{idn}.

\begin{table}[htb]
 \centering
  \begin{tabular}{ccc}
    \toprule
    タイプ & 値 & 目的 \\
    \midrule
    A & 1 &  ホストのIPv4アドレス \\
    NS & 2 & 権威サーバ \\
    MF & 4 & メール転送サーバ \\
    CNAME & 5 & 別名 \\
    SOA & 6 & 権威ゾーンの開始 \\
    NULL & 10 & NULL(実験用) \\
    PTR & 12 & ドメイン名のポインター(逆引き) \\
    HINFO & 13 & ホスト情報 \\
    MINFO & 14 & メールボックスおよびメールリスト情報 \\
    MX & 15 & メール交換 \\
    TXT & 16 & 任意文字列 \\
    \bottomrule
  \end{tabular}
 \caption{主要リソースレコード一覧}
 \label{tab:resource-record}
\end{table}

\newpage
ドメイン名に関連づけられる情報はリソースレコードと呼ばれ,目的に応じて複数のタイプが定義されている.
例えば,ドメイン名にIPv4アドレスを関連づけることを考える.
そのドメインの権威サーバは,関連づけたいIPv4アドレスをレコードタイプAとして,ゾーンファイルに記述する.
クライアントは,ドメイン名とレコードタイプとしてAを指定しサーバに問い合わせることで,権威サーバが事前に登録したAレコードに記述されたアドレスを取得することができる.
DNSでは,IPv4アドレス以外にも様々な情報をドメイン名に関連づけることができる.
代表的なソースレコードのタイプを表~\ref{tab:resource-record}で示す.

次に,DNSのパケットフォーマットについて説明する.
図~\ref{fig:dns-format}で示すように,DNSのパケットは5つのセクションに分けられる.
\begin{figure}[bh]
 \centering
 \includegraphics[scale=0.7]{figure/dns-format.png}
 \caption{DNSのパケットフォーマット}
 \label{fig:dns-format}
\end{figure}
クライアントがDNSの名前解決をする際,解決したいレコードタイプとそのドメイン名は図~\ref{fig:dns-answer}におけるQnameフィールに含まれる.
各フィールドはサイズが決まっており,Qnameフィールドが最大長255バイト,リソースレコードのタイプを表すQtypeフィールドとクエリクラスを表すQclassフィールドがそれぞれ2バイトとなっている.
\begin{figure}[th]
 \centering
 \includegraphics[scale=0.6]{figure/dns-answer.png}
 \caption{DNSのパケットのAnswerセクション(bytes)}
 \label{fig:dns-answer}
\end{figure}

\subsubsection{名前解決の仕組み}
\label{sec:dns-mechanism}

参考として,クライアントが``www.example.com"のIPv4アドレスを解決する場合を考える.
\begin{figure}[H]
 \centering
 \includegraphics[width=12.0cm]{figure/dns-name-resolution.png}
 \caption{DNSによる名前解決}
 \label{fig:dns-name-resolution}
\end{figure}
はじめに,クライアントとなるスタブリゾルバは,スタブリゾルバと同一セグメント内のフルサービスリゾルバもしくは,ネットワークセグメントに依らないどこからでもアクセスできるフルサービスリゾルバ(オープンリゾルバ,パブリックリゾルバとも呼称される)に問い合わせる.
フルサービスリゾルバは,その名前解決クエリが過去に解決したものでないかキャッシュデータを確認する.
キャッシュにヒットした場合にはキャッシュの情報をクライアントに応答され,ヒットしなかった場合には,root.hintsファイルを参照しルート権威サーバにリクエストパケットを転送する.
クエリ(問い合わせ)を受け取ったルート権威サーバは,``com"ドメインを委譲した権威サーバのアドレスを応答する.
次に,フルサービスリゾルバは,``com"の権威サーバに対し同じクエリを転送する.
``com"の権威サーバは,``example.com"ドメインを委譲した権威サーバのアドレスを応答する.
フルリゾルバは,``example.com"の権威サーバに同じクエリを転送する.
``example.com"の権威サーバは,保持するゾーンファイルからクエリされたドメインのリソースレコードについて探索し,探索の結果としてレコード情報をフルサービスリゾルバに応答する.
フルサービスリゾルバは,権威サーバから応答された情報をスタブリゾルバに転送することで,問い合わせられた名前は解決される.
DNSによる名前解決の一連の動作を図~\ref{fig:dns-name-resolution}で示す.


%リソースレコードのタイプごとの使用頻度を知りたい
% タイプごとの説明を充実させるのは,重要かもしれない


\newpage
\subsection{DNSトンネリング}
\label{sec:dns-tunneling}
%DNSトンネリングが脅威となりうる点に関する説明
%Botnetなどに使用されることについて言及するべき
%概念
現在の情報流出を目的としたマルウェアのほとんどは,秘匿通信手法を利用している~\cite{asaf}.
DNSトンネリングは,そのような秘匿通信の代表的な手法である.
DNSトンネリングは,DNSをデータ転送のメディアとした秘匿通信手法の総称であり,転送元と転送先の方向によって二つに分類することができる.
スタブリゾルバから権威サーバへの通信のDNS Exfiltrationと,権威サーバからスタブリゾルバへの通信のDNS Infiltrationである.
%DNSトンネリングという手法は,ポートスキャンで知られるNmapのメーリングリストだとされている.
DNSトンネリングは,以下に示すDNSの特性に基づいる.

\begin{itemize}
 \setlength{\itemsep}{0pt}
 \item 通常のインターネットの利活用において名前解決は必要な機能であるため,一般にDNSのサービスポートが閉ざされることがない
 \item 名前解決のトラフィックはほとんどのサービスに先立って発生するため,クエリログが肥大化しやすく長期のログ保存が困難である
 \item パケットフォーマットの構造において,任意の文字列を注入できるフィールドを保持する
\end{itemize}

DNSトンネリングがデータ転送のキャリアとするフィールドは,クエリのQuestionセクションのQnameと,AnswerセクションのRdataである.
QuestionセクションのQnameフィールドを利用することで,スタブリゾルバから権威サーバ方向にデータを転送できる.
この方向の通信は,ビーコン通信やターゲットから取得した情報を外部に漏えいさせるといった攻撃の最終目的を達成するのに使われる.
また,AnswerセクションのRdataフィールドを利用することで,データを転送することができる.
この通信は,ターゲットネットワーク内のホストに潜伏したマルウェアなどへの命令コードを送信するのに使われる.
さらに,この二つのキャリアを利用することが双方向の通信路を確保できるため,C2通信を実施することも可能である.
%DNSトンネリング手法が脅威なのは,IDS・IPSなどの検知システムにフィルタリングされにくく,クエリ頻度を長期化させた場合解析を迂回することができる点である.

DNSトンネリング手法が初めて一般に公開されたのは,1998年に,ポートスキャンで知られるNmapのメーリングリストだとされている~\cite{nmap, maarten}.
2004年には,Dan KaminskyがOzymanDNS~\cite{ozymandns}と呼ばれるDNSトンネリングの実装を公開した~\cite{dan-kaminsky}ことをきっかけに広く知られるようになった.
それ以降,数多くのDNSトンネリングの実装~\cite{heyoka, iodine, dnscat2, tcp-over-dns, dnscat, denise, dns-shell, dnsbotnet, dnscapy, dohtunnel, godoh, dohc2, magictunnelandroid, dns2tcp, tuns}が公開され,実際のサイバー攻撃に悪用されるようになっている.

\subsubsection{DNS Exfiltration}
\label{sec:dns-exfiltration}
% トンネリング実装のUpstreamは,どのくらいなのだろうか
\begin{figure}[bhtp]
 \centering
 \includegraphics[scale=0.6]{figure/dns-exfiltration.png}
 \caption[DNS Exfiltrationの概略図]{DNS Exfiltrationメソッドに基づいて,ドメイン名のラベル部に任意文字列がサブドメインとして注入されたDNSクエリが,イントラネット内のスタブリゾルバからインターネット上の権威サーバ(``exfil.com")に転送される様子}
 \label{fig:dns-exfiltration}
\end{figure}

本項では,スタブリゾルバから権威サーバ方向にデータを転送する手法であるDNS Exfiltrationの詳細について説明する.
DNS Exfiltrationは,名前解決として問い合わせられるドメイン名が,そのドメインのゾーンを管理する権威サーバに転送される仕組みを利用した手法である.
DNSでは,ドメイン名に関連づけられるリソースレコードの情報は,そのドメインをゾーンとする権威サーバが保持しており,ルートから再帰的に問い合わせていくことでその権威サーバからの応答を受け取る.
このため,問い合わせられたドメイン名が実在しない場合でも,再帰問い合わせの仕組みに従って,そのドメイン名の最後の権威サーバまで転送されることになる.
権威サーバでは通常,クエリされたドメイン名の実在有無に寄らず,問い合わせられたクエリ情報をログとして管理する.
このような特性に踏まえてDNSを利用すると,DNSクエリのドメイン名のラベルに組織外ネットワークに転送したい文字列を注入することで,組織外ネットワーク上に設置された権威サーバにそのデータを転送することができる.
これがDNS Exfiltrationの動作原理である.

このような仕組みのDNS Exfiltrationを動作させるには,宛先となる権威サーバを用意する必要があり,グローバルなドメインを取得することを前提としている.
第~\ref{sec:dns-mechanism}項で述べるように,ドメイン名の最大長は253バイトであり,その内ラベルの最大長は63バイトまでという制約がある.
そのため,DNS Exfiltration手法を用いてデータを転送する際には,TLDのラベルと宛先権威サーバのラベルもしくはSLDラベルと権威サーバのラベルを差し引いたサイズが実際に転送できる最大長となる.
また,任意の文字列をDNS Exfiltrationメソッドを用いて外部に転送するにあたり,転送キャリアであるドメイン名における文字列制約を満たすように転送したいデータに前処理を施す必要がある.
ドメイン名に使用できる文字列は,第~\ref{sec:dns-mechanism}項で述べるように,``a"から``z"までのアルファベットと``0"から``9"までの数字と先頭以外のハイフン``-"記号である.
この文字列制約については,転送したいデータをバイナリデータに変換し,そのバイナリデータをラベルとして印字可能なASCIIコードに変換することでその制約を満たすことができる.
この前処理について,既存のDNSトンネリング実装の多くがBase Encoding~\cite{rfc4648}を用いている.
%使用するエンコーディング手法によって,データの圧縮率は異なる.
この処理によって,転送データがバイナリデータである際にも転送効率上げたり,ラベルの文字列制約を満たさないデータも転送することができる.
また,エンコーディングのラベルは,自然言語とは異なるため,メッセージの意味抽出を困難にすることにも機能する.

ここで,DNS Exfiltrationを用いて,あるイントラネット内のホストからイントラネット外のホストにデータを転送することを考える.
転送される宛先となるイントラネット外のホストには,``exfil.com"より下位の全ての名前空間をゾーンとする権威サーバ(``exfil.com")を指定する.
転送したい文字列にエンコーディング前処理を施した後,``用意した文字列.exfil.com"という具合に文字列をラベルとして含めることで,ドメイン名が用意できる.
適当なリソースレコードタイプを指定し,DNSクエリとして転送すると,その権威サーバにはログとして,文字列を含んだドメイン名を取得する.
最後に,受け取ったサーバサイドは,前処理と逆のデコード処理を施すことで,オリジナルのデータを取得できる.
以上のように再帰問い合わせとラベルという転送キャリア,エンコーディング処理を組み合わせることで,イントラネット内のホストから外部ネットワークに任意の情報を転送することができる.
これが,DNS Exfiltrationの動作メカニズムである.
図~\ref{fig:dns-exfiltration}に,DNS Exfiltrationのメカニズムを図解した様子である.

%具体的な脅威モデル
%検知バイパス手法 : スループット(パフォーマンス)を下げることによる秘匿性,一般的なホスト名を使った対応表

%1998年4月,DNSトンネリングの手法は,NmapのBugtraqメーリングリストにて初めて公になったとされている\cite{bugtraq}.

\subsubsection{DNS Infiltration}
\label{sec:dns-infiltration}

\begin{figure}[bhtp]
 \centering
 \includegraphics[scale=0.6]{figure/dns-infiltration.png}
 \caption[DNS Infiltrationの概略図]{TXTレコードに登録された情報について,DNSクエリで問い合わせることで権威サーバから命令情報を取得している様子}
 \label{fig:dns-infiltration}
\end{figure}

本項では,権威サーバからスタブリゾルバ方向にデータを転送する手法であるDNS Infiltrationの詳細について説明する.

DNS Infiltrationは,DNSにおける幾つかのリソースレコードが任意の文字列を記述できる設計を利用したデータ転送手法である.
ドメイン名に関連づけられた情報を管理・提供する権威サーバは,ゾーンファイルに関連づけたい情報を記述する.
リソースレコードには,レコード情報を検証する機構が備わっていないため,任意の文字列を登録することができる.
特に,記法が決まっていないTXTタイプやNULLタイプなどもあり,DNS Infiltrationではこのようなレコードタイプに転送したいデータを登録しておく.
このようにして登録されたレコード情報について,名前解決問い合わせすることによって,インターネット(権威サーバ)からイントラネット(スタブリゾルバ)にデータを転送することができる.
DNS Infiltrationとして利用され得るレコードタイプについて,これまでのトンネリング実装で使用されたものに基づいてまとめたのが,表~\ref{tab:infil-rtype}である.
%zonefileを説明
%Infilとして使用される脅威のあるRtypeを列挙

\begin{table}[bh]
 \caption[DNS Infiltrationに使用されうるレコードタイプ]{DNS Infiltrationとして使用することができるレコードタイプの一覧}
 \centering
  \begin{tabular}{lrll}
    \toprule
		\multicolumn{1}{c}{\textbf{タイプ}} & \multicolumn{1}{c}{\textbf{\begin{tabular}{c}最大サイズ\\(byte)\end{tabular}}} & \multicolumn{1}{c}{\textbf{説明}} &\multicolumn{1}{c}{\textbf{実装}}\\
    \midrule
		A & 4 & \ ホストのIPv4アドレス &\\ \hline
		NS & 4 & \ 権威サーバ & \, \cite{dnscat2}\\ \hline
    %MF & 4 & メール転送サーバ \\
		CNAME & 253 & \ 別名 & \, \cite{iodine},\cite{dnscat2}, \cite{dnscapy}, \cite{tuns}\\ \hline
		%SOA & 253 & 権威ゾーンの開始 & \\
		NULL* & 255 & \ NULL(実験用) & \, \cite{iodine}\\ \hline
		PTR & 4 & \begin{tabular}{l}ドメイン名のポインター\\(逆引き)\end{tabular} & \\ \hline
    %HINFO & 13 & ホスト情報 \\
    %MINFO & 14 & メールボックスおよびメールリスト情報 \\
		MX & 253 & \ メール交換 & \, \cite{iodine},\cite{dnscat2}\\ \hline
		TXT* & 255 & \ 任意文字列 & \begin{tabular}{l}\cite{iodine},\cite{dnscat2}, \cite{denise}, \cite{dns-shell},\\ \cite{dnscapy}, \cite{dohtunnel}, \cite{dohc2}\end{tabular}\\ \hline
		AAAA & 32 & \ ホストのIPv6アドレス & \\ \hline
		SRV & 180 & \begin{tabular}{l}ドメイン名に対する\\サービスの場所\end{tabular} & \, \cite{iodine}\\ \hline
		 DNSKEY & 40 & \ DNSSECのための公開鍵 & \, \cite{dns2tcp}\\
    %TLSA & 52 & \\
    \bottomrule
  \end{tabular}
 \label{tab:infil-rtype}
\end{table}

NS・CNAME・MXレコードでは,DNS Exfiltrationと同じ要領でドメイン名のラベルに転送したい文字列を注入できる.
また,NULL・TXT\footnote{EDNS0を使用する場合,65535bytesが最大長となる.}・SRV・DNSKEYを用いる場合には,レコード構文に指定がないため任意の文字列をそのまま注入できる.
最後に,A・AAAA・PTRレコードを用いる場合には,転送したい文字列を数字に変換させた後に,ドット(.)区切りもしくはコロン(:)区切りで注入できる.

TXTレコードタイプを用いてDNS Infiltrationすることを考える.
(\ref{eq:infil-txt})は,TXTレコードを用いてホストで実行させるLinuxコマンドを転送する例である.
転送されるunameプログラムは,システム情報を取得するプログラムである.
実行結果をDNS Exfiltration手法を用いて転送することで,ホストマシンにおけるシステムのカーネルバージョンやプロセッサの種類などの情報を取得することができる.
\begin{eqnarray}
 www.exfil.com. \qquad IN \quad TXT \quad uname \ -ap
 \label{eq:infil-txt}
\end{eqnarray}
次に,スタブリゾルバは,``www.exfil.com"の``TXT"レコードタイプを通常通り問い合わせる.
再帰問い合わせの仕組みに基づいて,そのDNSクエリは``exfil.com"まで転送され,ゾーンファイルのTXTレコードタイプの値がフルサービスリゾルバを経由したのち,スタブリゾルバまで応答される.
DNS Infiltrationの流入通信を図解した様子が,図~\ref{fig:dns-infiltration}である.
このようにして,DNS Infiltrationでは,正規の名前解決の方法を用いて,インターネットから組織内へとデータを取得することができる.

% Null タイプは,厳密に定義されておらず,実験用としか表現されていない.しかし,全体のタイプのうち,20%を示す程度に頻繁に使用されるタイプのである.

\newpage
\subsection{DNSトンネリングへの既存対策}

本節では,DNSトンネリングに対するこれまでの対策手法を紹介し,検知手法として用いられるアプローチについて説明する.
最後に,それら検知アプローチを迂回する脅威モデルを示し,検知に基づくアプローチに限界があることを明らかにする.

DNSトンネリングに対して,森下らは提案されているDNSトンネリングへの対策アプローチを以下のようにまとめている~\cite{morishita}.
\begin{enumerate}
 \item DNSクエリログの取得と保存・内容の調査
 \vspace{-0.5cm}
 \item エンタープライズネットワークにおけるOP53B\footnote{OP53B : 組織外に設置されたオープンリゾルバのようなフルサービスリゾルバや権威サーバとの通信を抑止するために,組織外ネットワークを宛先とする53番ポートの通信をブロックする仕組み}の適用
 \vspace{-0.5cm}
 \item DNSファイヤーウォールの導入
\end{enumerate}

クエリログの取得は,リアルタイムではなく後に解析・調査するにためには必要不可欠である.
OP53Bの適用は,組織におけるDNS通信をイントラネットのフルサービスリゾルバに集約するのに効果が期待される.
DNSファイヤーウォールの導入は,ベンダーが開発したトンネリング通信のモデルや閾値に基づき検知またはブロックするのに効果が期待される.

% 検知に基づく手法の現在までの程度を淡々と示す.
%誰がその特徴を発見したのか
%それ特徴を用いて,どのようなテクニックで検知手法として確立させたのか
\subsubsection{特徴量}
\label{sec:pre-tunnel-feature}
従来,DNSトンネリング通信の検知には,以下に示す特徴量について統計分析を用いた閾値の算出や機械学習を用いた悪性モデルが用いられてきた.
トンネリング実装などによって発生する一般的なDNSトンネリング通信の検知にあたり,以下のような特徴を利用した手法がこれまでに多数提案されている.\newline
%長さ
%DNS Exfiltrationでは,Qnameフィールドのラベルがデータ転送のキャリアとなるため,表~\ref{tab:feature-tunnel}で示すように,転送するデータ量に比例してドメイン名は長くなる.
%また,DNS Infiltrationの場合は,応答パケットのRdataフィールドがデータ転送のキャリアとなるため,転送するデータ量に比例して肥大化する.


\hspace{-12pt}\textbf{ドメイン名の長さとクエリパケットのサイズ}\\
\hspace{12pt}クライアントからサーバ方向にデータを転送させるDNS Exfiltration手法において,転送キャリアとなるドメイン名が注入されるデータ量に応じて長くなる~\cite{vern}.
例えば,DNS Exfiltrationにおいて,一回あたりのデータ転送量を増加させる場合,Qnameフィールドのドメイン長もそれに比例して長くなり,結果としてパケットサイズも増加するという具合である.
DNS Infiltrationにおいても同様で,一回あたりのデータ転送量を増加させる場合,Rdataフィールド内のデータ量も大きくなり,応答パケットのサイズが増加する.

\begin{table}[th]
 \caption{正規DNSクエリとDNSトンネリングにおけるドメイン名の違い}
 \centering
  \begin{tabular}{l|l}
    \toprule
		\multicolumn{1}{c}{\textbf{種類}} & \multicolumn{1}{c}{\textbf{ドメイン名}} \\
    \midrule
    正規 &  www.example.com \\ \hline
    トンネリング & arbitrary-text.you-can-input-here-as-label.example.com\\
    \bottomrule
  \end{tabular}
 \label{tab:feature-tunnel}
\end{table}


%同一ドメインあたりのクエリ頻度
\hspace{-12pt}\textbf{同一ドメインあたりのトラフィック頻度}\\
\hspace{12pt}DNSトンネリングでは,一度に転送できるデータ量に限界があるため,目的のデータを全て転送するには分割する必要がある.
トンネリング実装のように対話的にシェルコマンドを実行する通信の場合,トラフィック頻度は極めて高頻度になる.
また,サイズの大きいデータをDNS Exfiltrationを用いて転送する場合も同様に,複数のパケットに分割されたデータを転送するにあたって,多数のトラフィックが発生することになる.\newline

%レコードタイプ
\hspace{-12pt}\textbf{リソースレコードのタイプ}\\
\hspace{12pt}理論的に全てのリソースレコードを用いてデータを転送することは可能であるが,第~\ref{sec:dns-infiltration}項で示すように,使用するレコードタイプによって転送できるデータ量は大きく異なる.
表~\ref{tab:infil-rtype}で示すように,実際のトンネリング実装におけるDNS Infiltarationを目的とする通信では,AやAAAAなどは使われず,CNAMEやTXTが主に使用される.
AやAAAAなどのレコードタイプが使用されない背景には,数字のみの文字列制約が厳しさと最大のデータサイズに小さいことが考えられる.
TXTの最大サイズが253bytesであるのに対して,Aが4bytesでAAAAが32bytesなのは明らかに小さいことが確認できる.
他方で,表~\ref{tab:distribution-rtype}で示すように,通常のインターネットの利活用において使用されるレコードタイプに極端な分布の偏りがあることが知られている.
Herrymannら~\cite{Herrymann}は,2010年1月1日から6月30日までの期間において,大学構内に設置されたフルサービスリゾルバによるDNSログデータを収集した.
結果が示すように,ドメイン名に対するアドレス解決の通信が全体の89.407\%を占めていることが確認できる.
以上のことから,TXTやCNAMEといった任意の文字列を注入できるレコードタイプは効率的なデータ転送を実現できる反面,レコードタイプとして使用頻度低いという特性から,データ転送と秘匿性がトレードオフの関係にあることがわかる.

\begin{table}[htb]
 \caption{レコードタイプの分布}
 \centering
  \begin{tabular}{lrr}
    \toprule
    \multicolumn{1}{c}{\textbf{Type}} & \multicolumn{1}{c}{\textbf{パケット数}} & \multicolumn{1}{c}{\textbf{割合}} \\
    \midrule
    A & 236,210,050 & 54.778 \\
    AAAA & 149,322,427 & 34.629 \\
		PTR & 43,060,608 & 9.986 \\
		SRV & 1,497,622 & 0.347 \\
		MX & 474,827 & 0.110 \\
		ANY & 281,023 & 0.0657 \\
		SOA & 226,975 & 0.053 \\
		TXT & 115,300 & 0.027 \\
		NS & 12,028 & 0.003 \\
		TKEY & 4518 & 0.001 \\
		NAPTR & 4281 & 0.001 \\
		SPF & 512 & 0.000 \\
		CNAME & 196 & 0.000 \\
		AXFR & 2 & 0.000 \\
		NULL & 2 & 0.000 \\ 
		合計 & 431,210,371 & 100.000 \\
    \bottomrule
  \end{tabular}
 \label{tab:distribution-rtype}
\end{table}


%これは,新規登録のドメイン名のみしか対応していないようだ
%Tatangら~\cite{tatang}が,2017年7月30日から9月1日までの期間にて,DNSサーバ間の通信パッシブDNS
%
%\begin{table}[h]
 \caption[リソースレコードの分布]{2017年7月から8月までのDNSトラフィックデータセットにおけるリソースレコードのタイプ分布}
 \centering
  \begin{tabular}{lrr}
    \toprule
		\multicolumn{1}{c}{\textbf{タイプ}} & \multicolumn{1}{c}{\textbf{パケット数}} & \multicolumn{1}{c}{\textbf{割合(\%)}}\\
    \midrule
    A & 1,121,025,638 & 54.90\\
    AAAA & 197,388,865 & 9.67\\
    MX & 682,948 & 00.3\\
    NS & 7,662,147 & 0.38 \\
    CNAME & 156,708,021 & 7.68 \\
    TXT & 41,593,164 & 2.04 \\
    NULL & 432,232,574 & 21.17 \\
    Other & 84,371,709 & 4.13 \\
    \bottomrule
  \end{tabular}
 \label{tab:infil-rtype}
\end{table}

%による調査では,表~\ref{tab:distribution-rr}で示す通り
\newpage
\hspace{-12pt}\textbf{パケットの応答ステータス}\\
\hspace{12pt}DNSのヘッダーは,図~\ref{fig:dns-header}で示すようなフィールドを持っており,問い合わせに対して,表~\ref{tab:dns-rcode}のようなステータス情報を応答する.
第~\ref{sec:bypass-method}項で示すような検知迂回手法を使う場合を除いて,
通常のDNS Exfiltrationでは,権威サーバが未知のデータがクラアントから転送される.
そのため,クライアントからの問い合わせには,コンテンツ不在を意味する``NXDomain"が応答される.
応答パケットのステータスが``NXDomain"であるとき,DNS Exfiltrationの可能性がある.
%表に書き換えることが可能である
\begin{figure}[h]
 \centering
 \includegraphics[width=10.0cm]{figure/dns-header.png}
 \caption{DNSのヘッダー(bytes)}
 \label{fig:dns-header}
\end{figure}

\begin{table}[h]
 \caption{代表的なRcode一覧}
 \centering
  \begin{tabular}{rll}
    \toprule
		\multicolumn{1}{c}{\textbf{値}} & \multicolumn{1}{c}{\textbf{名前}} & \multicolumn{1}{c}{\textbf{意味}}\\
    \midrule
    0 & NoError & 正常\\
    1 & FormErr & フォーマットエラー\\
    2 & ServFail & サーバエラー\\
    3 & NXDomain & 存在しないドメイン\\
    4 & NotImp & 未実装\\
    5 & Refused & 問い合わせ拒否\\
    \bottomrule
  \end{tabular}
 \label{tab:dns-rcode}
\end{table}


%Qnameにおける文字列分布とエントロピー
% 結局Baseエンコーディングは,単に文字列をマッピングさせているだけなので,一方向性ではないので,オリジナルの文字列の分布と変化はない.
\hspace{-12pt}\textbf{ドメイン名に含まれる文字列の出現頻度}\\
\hspace{12pt}Bornら~\cite{born}は,ドメイン名に使用されている文字列の分布について,流布しているトンネリング実装と正規のDNS通信について調査した.
その結果,正規のドメイン名が英語における文字列の出現分布と相関があるのに対して,トンネリング実装によって生成されるドメイン名における文字列の出現頻度では相関がみられず,文字列の出現頻度はランダムとなる傾向にある.

%どのようめ手法を用いて,DNSトンネリング検知に取り組んできたのかを説明し,次の検知迂回手法の足がかりにする
%例えば,クラスタリングなのか,機械学習なのか,統計的な手法なのか
%ツール依存であり汎用性がないことを述べたい
%これまでに提案されてきたDNSトンネリングの検知手法は,第~\ref{sec:pre-tunnel-feature}項で述べるような特徴量に基づいている.


%全セクションで述べられるような検知手法を迂回する手法を述べ,既存の検知手法の限界を示し,アーキテクチャの変更の必要性を論じる
\subsubsection{検知迂回の脅威モデル}
\label{sec:bypass-method}
%パフォーマンスを下げる手法(Low Latency)
%DNSエイリアスに対応づける手法
% 慣習的に命名されるラベル(Naming Convention)に意味を持たせることによって,例えば1byteの情報を持たせること
%特別Alexaなどの人気ラベルを調査しなくてもいいのかもしれない
% 既存研究における課題として,ツール限定型の手法では汎用性がない
% しかし,これまでに提案されてきた手法は,Low throughput手法に対して十分ではない機能しない可能性がある
第~\ref{sec:pre-tunnel-feature}項で述べる特徴量に基づき通信を監視することで,Iodineに代表されるトンネリング実装の通信を検知するのは,既存の検知手法で十分対処することは可能であると考えられる.
しかし,秘匿性を高めたDNSトンネリング通信に対しては,既存の検知手法では十分でない場合が考えられる.
例えば,DNSトンネリングの秘匿性を高める手法には,一回の問い合わせで転送するデータサイズを小さくする方法が考えられる.
従来の検知手法では,通常のDNS通信の統計的な分布から外れる異常通信として検知する.
データサイズを小さくするということは,一般のDNSクエリに使用されるようなドメイン長や応答パケットサイズに調整するということである.
この場合,従来のトンネリング実装などと比べて転送効率は下がる代わりに,パケットは一般のトラフィックに極めて類似することができる.

このように異常通信の特徴になる要素を削減することで,通常のインターネットの利用時に見られる入力ミスによって生じる通信に分類される通信クラスに模倣するというものである.
さらに,トラフィックの頻度の抑える手法を組み合わせることが考えられる.
第~\ref{sec:pre-tunnel-feature}項で述べるようにトラフィックの頻度に基づいて異常を検知するアプローチは広く利用されているが,DNSの通信においてはトラフィック量が肥大化しやすく長期間ログが取得されることは稀であることが予想される.
APTに代表される攻撃では,攻撃手法の解析などを回避するために,極めて高いモチベーションでこのような秘匿技術が用いられることが予想される.
現在の検知に基づく対策では,このような秘匿通信に対処するには課題がある.
Asafらは,システムのメモリやディスクを追加の特徴量とすることによって,そのような秘匿性を高めたトンネリング通信を検知する手法を提案している~\cite{asaf}.
しかし,Asafらの提案手法では,リアルタイム性が考慮されていない.
攻撃者にとって,秘匿手法によって小分けされたデータが情報という意味のあるものになったタイミングで情報流出という目的は達成される.
すなわち,情報流出の本質的な対策には,検知に基づく事後対応ではなく未然に抑止することが求められる.

\section{関連研究}
\label{sec:related-works}
本章では,はじめに既存のDNS Tunnelingに対するアプローチとして提案されている検知アプローチを取り上げ,現在の検知に基づく対策の課題として,Low Throughput手法とSlowな転送手法というバイパス手法に対処できないことを明らかにする.
次に,これまでに提案されてきたP2Pベースの名前解決システムを説明し,提案手法との違いを示す.
%また,新しいアーキテクチャを導入するとき 既存のシステムとのマイグレーションを考慮する必要がある点について,本提案手法がマイグレーションを考慮している設計である点について
%最後に,既存の検知手法および次世代名前解決システムの課題から,既存のシステムに迎合しながらDNS Tunnelingを緩和する名前解決システムの必要性を明らかにする.

\subsection{DNS Tunnelingの検知手法}
\subsubsection{特徴量に基づく閾値推定}
\subsubsection{機械学習に基づくモデル生成}
%\subsubsection{パターンマッチング}
%\subsubsection{同一ドメインあたりのクエリ頻度}
%\subsubsection{Qnameにおける文字列分布}
%\subsubsection{Qnameにおける長さとエントロピー}
%\subsubsection{課題 : Low ThroughputなTunnelingに対する検知手法}
DNS Tunnelingメソッドを使用した時のDNSクエリは,\ref{sec:dns-tunnel}で述べたような特徴が現れる性質がある.
この性質に基づき,これまでに多数の検知手法が提案されてきた.

Bornら\cite{born}は,

%\subsection{悪性DNS検知に関する研究}
\subsection{アーキテクチャに基づく抑制手法}
\subsubsection{P2Pベース名前解決システム}
これまでに,DNSにおける〜の課題に対して,P2Pに基づいた名前解決システムは数多く提案されてきた.
\subsubsection{Blockchainベース名前解決システム}
\subsection{課題}

\section{提案システム}
本章では,DNSトンネリングの発生抑止を目的とした名前解決システムDNS-TD(DNS for Tunneling Deterrence)について説明する.
%第\ref{sec:related-works}で述べたように,これまでに提案されてきた検知に基づくDNS Tunneling対策には,Low Throughput手法および転送頻度を下げる手法に対して,検知が困難であるという課題がある.
%他方,新しいアーキテクチャに基づく名前解決システムには,マイグレーションの課題が残留している.
\begin{figure}[h]
 \centering
 \label{fig:abstruct-DNS-TD-architecture}
 \includegraphics[scale=0.6]{figure/new-architecture-DNS-TD.png}
 \caption{提案システムの概略図}
\end{figure}

\subsection{概要}
\label{sec:DNS-TD}
%現在の課題
%目的をしっかりと説明すること
%現在使用されているDNSに基づく名前解決において,問い合わせのパケットを悪用することで任意のデータを送受信するキャリアとして悪用するDNSトンネリングという秘匿通信手法として利用される設計上の不備がある.
%検知に基づいた対策アプローチが過去に多数提案されているが,転送量および頻度を調整する迂回手法を用いる場合には対処するのに不十分であるという課題がある.
%DNSトンネリングは,クライアントからの名前解決クエリがコンテンツを保持するサーバに透過的に転送されることに起因して発生する.
%提案システムでは,この点に着目し,
%%そこで,DNSトンネリングが発生しない
%この課題に対して,トンネリング通信を抑止することを目的とする新しい名前解決システム(DNS-TD)を提案する.
%提案システムでは,クライアントからの名前解決問い合わせがコンテンツの操作と保持の機能を兼ねるサーバに転送されることに着目し,その機能を分離させることによって,トンネリング抑止を実現する.
%%フラットな名前空間と分散ハッシュテーブルの仕組みを利用することによるトンネリング通信の発生を抑止する名前解決システムを提案する.
%%設計要件
%提案システムのDNS-TDの主要な要素は,以下の通りである.
%\begin{itemize}
%% \item 84byteの名前空間における全てのコンテンツ情報をマップ
%% \vspace{-3mm}
% \item ゾーンをハッシュ空間の範囲に基づき分割
% \vspace{-3mm}
% \item 権威サーバにおけるレコード情報の操作機能と管理機能の分離
% \vspace{-3mm}
% \item ドメイン名ではなく識別子に基づく名前解決
% \vspace{-3mm}
% \item レコード情報に対する認証機能
%\end{itemize}
%ハッシュ値に基づくフラットな名前空間をソートした範囲に基づき分割し,コンテンツを保持する管理サーバによって分散的に管理する.
DNS-TDでは,クライアントからの名前解決クエリを処理するサーバは,コンテンツを操作する機能を持たないノードによって処理される設計になっている.
コンテンツ情報を変更する機能とクライアントにコンテンツを提供する機能を分離することによって,DNSトンネリングにおけるクライアントとコンテンツを操作するサーバとの通信の発生を抑止する.
また,DNS-TDでは,ゾーンファイルによって権威サーバによってドメインに紐付けるレコード情報を一元的に管理するのではなく,リソースレコードのタイプごとにコンテンツの識別子を付与し,識別子に基づき担当のサーバ,提案システムではマネージャによって分散的に管理される.
識別子は,84bytesの名前空間をもつハッシュ関数によって算出されるメッセージダイジェストを利用する.
レコード情報を操作するのは,実際にレコード情報を保持するマネージャではなく,マネージャに階層的に連結したプロバイダと呼ばれるサービスノードである.
マネージャは,プロバイダからのレコード情報に対する操作リクエストに基づいて,担当のマネージャに転送する.
DNS-TDの名前空間は,ソートされたハッシュ値の範囲に基づき分割され,既存システムのTLDによって分割された範囲をゾーンとして分割的に管理される.
DNS-TDにおけるクライアントは,シンボルを算出することによって,レコード情報を保持するノードを一意に特定することができる.
また,シンボルからレコード情報を保持するノードを一意に特定の名前解決プロセスでは,レコード情報を作成したエンティティは介在されない.
すなわち,レコード情報を作成する機能とレコード情報を管理する機能を異なるサービスノードに分離させることによって,DNS-TDではDNS Exfiltrationが発生することを抑止する.

DNS-TDでは,DNS Infiltrationに対して,認証基盤の導入と使用できるリソースレコードを制限するメソッドを採用する.
認証基盤では,ドメインに関連づけるレコード情報との関連性を評価することで不審なレコード情報がハッシュテーブル上にストアされることを防止する.
Qnameにシンボルとして84bytesのメッセージダイジェストを使用することによって,副次的効果として,偽装DNS応答パケットの作成が困難にできるという性質が期待される.
また,既存システムにおけるDNSSECによって実現されてきた送信元のトレーサビリティについて,DNS-TDでは認証基盤とシンボルに基づくDNS応答パケットの偽装困難性によって実現され,DNSSECの必要性はなくなる.
さらに,任意の文字列を注入することができるレコードタイプNULLとTXTについて,実験目的のNULLタイプの使用制限とTXTタイプは機能をシンタックスの限定しているSPFに回帰させることで対処する.
DNS-TDにおける用語については,表~\ref{tab:refres-terminology}で示す.
\begin{table}[p]
 \caption{DNS-TDにおける用語}
 \centering
  \begin{tabular}{ll}
    \toprule
    \multicolumn{1}{c}{\textbf{表記}} & \multicolumn{1}{c}{\textbf{意味または機能}}\\
    \midrule

    コンテンツ & \begin{tabular}{l}・識別子に関連づけられたレコード情報の実体\end{tabular}\\ \hline

    コンテンツID & \begin{tabular}{l}・識別子\end{tabular}\\ \hline

    ドメインID & \begin{tabular}{l}・識別子(コンテンツIDが重複した際に使用)\end{tabular}\\ \hline

    レコード情報 &
      \begin{tabular}{l}
        ・リソースレコードの具体的な値\\
        $\,$ (例 IPアドレス)
      \end{tabular}\\ \hline

     リソースレコードタイプ &
      \begin{tabular}{l}
        ・オブジェクトに関連づけるリソースレコードの型\\
        $\,$ (例 A, AAAA, MX)
      \end{tabular}\\ \hline

    オブジェクト &
      \begin{tabular}{l}
       ・問い合わせる対象\\
       $\,$ (ドメイン名もしくはIPアドレス)
      \end{tabular}\\ \hline

    スタブリゾルバ & \begin{tabular}{l}・名前解決クライアント\end{tabular}\\ \hline

    フルサービスリゾルバ &
      \begin{tabular}{l}
       ・スタブリゾルバからのクエリハンドリング\\
       ・識別子の作成
      \end{tabular}\\ \hline

    マネージャ &
      \begin{tabular}{l}
       ・フルサービスリゾルバからのクエリハンドリング\\
       ・ゾーンの管理\\
       ・コンテンツの保持
      \end{tabular}\\ \hline

    プロバイダ & \begin{tabular}{l}・コンテンツの作成・更新・削除操作\end{tabular}\\

    \bottomrule
  \end{tabular}
 \label{tab:refres-terminology}
\end{table}


\newpage
\subsection{システムアーキテクチャ}
\begin{figure}[htbp]
 \centering
 \includegraphics[scale=0.5]{figure/manager-provider.png}
 \caption{マネージャとプロバイダの関係図}
 \label{fig:manager-provider}
\end{figure}

DNS-TDでは,全体のアーキテクチャを従来のクライアントサーバアーキテクチャを踏襲している.
クライアントサーバアーキテクチャを採用することによって,クライアントは既存のDNSサービスに変更を加えることなく使用することができる.
他方で,サーバ同士はフルメッシュなネットワークで構成される.
%なぜこのアーキテクチャにしたのか

\subsection{サービスノード}
DNS-TDを構成するサービスノードは,以下の4つである.
\begin{itemize}
 \item スタブリゾルバ
	\vspace{-3mm}
 \item フルサービスリゾルバ
	\vspace{-3mm}
 \item マネージャ
	\vspace{-3mm}
 \item プロバイダ
\end{itemize}

DNS-TDでは,権威サーバが機能に基づき二つサービスノードに分割される.
DNS-TDでは,既存のDNSにおける権威サーバの機能を二つのサービスノードで動作させることによって,DNSトンネリングの抑止を図る設計になっている.
権威サーバの機能を列挙すると以下のようになる.
\begin{itemize}
 \item 上位ドメインから委任されたゾーンを管理する機能
	\vspace{-3mm}
 \item レコード情報を保持する機能
	\vspace{-3mm}
 \item スタブリゾルバからの名前解決問い合わせに応答する機能
	\vspace{-3mm}
 \item レコード情報を作成・修正・削除する機能
\end{itemize}
DNSトンネリングは,これら機能と性質を組み合わせることで発生する手法である.
例えば,スタブリゾルバから権威サーバ方向のデータ転送手法であるDNS Exfiltrationは,``スタブリゾルバからの名前解決問い合わせに応答する機能"と``"を組み合わせることで機能する.
また,権威サーバからスタブリゾルバ方向のデータ転送手法であるDNS Infiltrationは,``スタブリゾルバからの名前解決問い合わせに応答する機能"と``ゾーン内のレコード情報を作成・修正・削除する機能"が組み合わさることで機能するとそれぞれ捉えられる.
そこで,DNS-TDでは,権威サーバの機能を二つのサービスノード(マネージャとプロバイダ)に分けることで,既存の名前解決機能とDNSトンネリング抑止機能を実現する.\newline

\hspace{-12pt}\textbf{スタブリゾルバ}\\
\hspace{12pt}スタブリゾルバは,既存のDNSと同様,ドメイン名に関連づけられたリソースレコード情報を問い合わせるクライアントである.
DNS-TDが,既存のシステムと大きく異なるのは,問い合わせられたドメイン名とそのレコードタイプからコンテンツIDを識別子とすることで,レコード情報を解決する点である.
これを実現するために,既存システムのフルサービスリゾルバと権威サーバに変更が加えられ,機能が分割している.\newline

\hspace{-12pt}\textbf{フルサービスリゾルバ}\\
\hspace{12pt}フルサービスリゾルバは,従来システムと同様のキャッシュ機能を担いながら,コンテンツIDの算出し,コンテンツを保持するマネージャに問い合わせるサービスノードとして機能する.
フルサービスリゾルバは,スタブリゾルバからの従来のフォーマットのDNSクエリについて,ドメイン名とレコードタイプからハッシュ値の算出を行う機能を担う.
\begin{algorithm}[h]
 \caption{フルサービスリゾルバにおける問い合わせ転送処理}
 \label{algo:full-service}
  \SetKwProg{Fn}{}{\string:}{}
  \SetKwFunction{Handle}{handler}
 $\vspace{-0.3cm}$\;
 クエリハンドリング\;
 \Fn{\Handle{query\_data,\ rtype}}{
   $content\_id,\ domain\_id \leftarrow calculate\_id(query\_data,\ rtype)$\;
	 $manager\_addr \leftarrow find\_manager(start,\ end,\ content\_id)$\;
	 $answer \leftarrow query\_manager(manager\_addr,\ content\_id,\ domain\_id)$\;
	 $response\_client(client\_address,\ qname,\ answer.rcode,\ answer.rdata)$\;
 }
 $\vspace{-0.3cm}$\;
\end{algorithm}


\newpage
\hspace{-12pt}\textbf{マネージャ}\\
\hspace{12pt} マネージャは,ドメイン名に対応するレコード情報を保持し,フルサービスリゾルバからの問い合わせに応答するサービスノードである.
マネージャは,既存のDNSにおけるTLDに相当する権威サーバが担当する.
%マネージャの数
%イベントドリブン
%非同期
%フルサービスリゾルバからのクエリがあった時の動作を参考として説明する
\begin{algorithm}[!h]
 \caption{マネージャにおける名前解決問い合わせ処理}
 \label{algo:manager}
  \SetKwProg{Fn}{}{\string:}{}
  \SetKwFunction{Handler}{handler}
  \SetKwFunction{Parse}{parser}
  \SetKwFunction{Database}{db\_accesser}
  \SetKwFunction{Noerror}{generate\_packet}
  \SetKwFunction{Error}{generate\_errror}
 $\vspace{-0.3cm}$\;
 %Calculate the content'{}s content id and domain id\;
 \Fn{\Handler{query\_data}}{
	 $content\_id,\ qtype \leftarrow parser(query\_data)$\;
	 $record\_value \leftarrow db\_accesser(content\_id)$\;
	 \If{$value$}{
		 $payload \leftarrow benigh\_response(content\_id,\ qtype,\ ttl,\ record\_value) $\;
		}
		\Else{
		 $payload \leftarrow error\_response(content\_id,\ qtype)$\;
		 }
		$payload \leftarrow payload.pack()$\;
		$sendto(payload,\ client\_address)$\;
 }

%クエリのパース\;
% %Calculate the content'{}s content id and domain id\;
% \Fn{\Parse{data}}{
%   $payload = DNSRecord.parse(data)$\;
%	 $return \ {'packet\_id':payload[0], 'content\_id':payload[1], 'q\_type':payload[2]}$\;
% }
%
%
% $\vspace{-0.3cm}$\;
% %Find the manager who has zone includes the content id\;
% DBへアクセス\;
% \Fn{\Database{content\_id}}{
%	$return \ Redis("127.0.0.1", 6379).get(content\_id)$\;
% }
% $\vspace{-0.3cm}$\;
%
% %Query the content to the manager\;
% 応答パケットの作成\;
% \Fn{\Noerror{packet\_id, content\_id, q\_type, ttl, record}}{
%	$payload = DNSRecord(DNSHeader($\;
%				$qr=1, aa=1, ra=1,id=packet\_id, rcode=RCODE["NoError"]))$\;
%	$payload.add\_question(content\_id, q\_type)$\;
%	$payload.add\_answer(c\_id, ttl, record)$\;
%  $return \ payload$\;
% }
% $\vspace{-0.3cm}$\;
%
% %Transfer the answer to client\;
% エラー応答パケットの作成\;
% \Fn{\Error{packet\_id, content\_id, q\_type, ttl, record}}{
%	$payload = DNSRecord(DNSHeader($\;
%				 $qr=1, aa=1, ra=1,id=packet\_id, rcode=RCODE["NXDomain"]))$\;
%	$payload.add\_question(content\_id, q\_type)$\;
%  $return \ payload$\;
% }
\end{algorithm}


\hspace{-12pt}\textbf{プロバイダ}\\
%マネージャとのどのような階層構造なのか
%どのようなプロトコルでパケットをマネージャに転送するのか
\hspace{12pt}プロバイダは,レコード情報を作成・更新および消去といった操作を担当するサービスノードである.
プロバイダは,既存のDNSにおけるSLD以降の権威サーバに相当し,ドメインの階層構造を従いマネージャに接続される.
プロバイダは,マネージャを介在することで,レコード情報を操作することができる.
例えば,example.comプロバイダが``www"のIPアドレス情報を作成することを考える.
example.comプロバイダは,``www.example.com"とレコードタイプ``A"およびその値``93.184.216.34"を含むデータを接続先のcomマネージャにリクエストする.
comマネージャは,リクエストされたドメイン名とそれに関連づけるレコードタイプから識別子を算出し,担当のマネージャにストアリクエストを転送するという具合で動作する.

\newpage
\subsection{識別子}
%ハッシュ空間によってIDが管理されることとどのハッシュ関数を使用するのかを説明する.
本節では,レコード情報にアクセスするために用いる識別子について説明する.
DNSにおける名前解決では,ドメイン名とレコードタイプの情報をサーバに問い合わせ,サーバは保持するゾーンファイルから該当するレコード情報が応答される仕組みになっている.
他方,DNS-TDでは,ドメイン名とレコードタイプに基づき算出されるコンテンツIDとドメインIDを識別子としてサーバに問い合わせ,サーバは識別子をキーとする対応するバリューを応答させることで名前が解決する仕組みになっている.
ドメイン名とレコードタイプの全ての組み合わせをフラットな名前空間上に対応づける設計において,名前空間の大きさは数の不足を無視できる程度に大きくなくてはならない.
既存のDNSのプロトコルフォーマットが広く一般化している点に留意し,提案システムではプロトコルへの変更を加えない設計デザインに思想としている.
このため,既存のDNSのQuestion Sectionなどにおけるサイズの制約を満たす必要がある.
ドメイン名が格納されるQnameは,255bytesである.
また,DNSのラベルの最大長は63bytesである.
この識別子は,ドメイン名とレコードタイプをメッセージとするハッシュ関数から算出されるメッセージダイジェストである.
コンテンツIDは,ドメイン名とレコードタイプの文字列の和を引数とするメッセージダイジェストである.
ドメインIDは,ドメイン名を引数とするメッセージダイジェストである.

%以上のことから,84bytesの名前空間をDNS-TDでは採用する.
% 84bytesの名前空間を持つこととダイジェストの長さは違う
% sha3_224のダイジェスト長は,56文字

\subsubsection{ハッシュアルゴリズム}
本項では,コンテンツIDとドメインIDに使用されるハッシュアルゴリズムについて詳しく説明する.
DNS-TDで使用するハッシュアルゴリズムには,以下のような制約がある.

\begin{enumerate}
 \item 名前空間は不足を無視できる程度に大きくなくてはならない
 \vspace{-3mm}
 \item アルゴリズムは一方向性の性質を備えなくてはならない
 \vspace{-3mm}
 \item ラベル長は最大63byte,ドメイン長は最大253byteである
 \vspace{-3mm}
\end{enumerate}

ドメイン名とリソースレコードタイプの組をハッシュ値の引数とするため,コンテンツIDは,レコード情報の数に比例して増加する特性がある.

また,識別子の引数の一つにドメイン名が含まれていることから,識別子から元のメッセージが導き出くことが困難な性質を備えていなくてはならない.
この性質を満たすことで,なんらかの方法で識別子を悪意の第三者が取得された際にDNS Exfiltrationとしてデータを転送されることを抑止することができる.
%ここで,既存のハッシュ関数の特性をまとめたtableを示す.
以上から,DNS-TDでは,56byteの名前空間をもつsha3のアルゴリズムを採用する.

メッセージダイジェストは,コリジョンが無視できる程度に大きな名前空間と分布を備えたハッシュ関数を備えなくてはならない.
以降では,コリジョンが発生した対処法として採用する分離連鎖法と2重ハッシュ法について説明する.
DNS-TDでは,コンテンツのストアリングフェーズでIDにコリジョンが発生した場合,分離連鎖法に基づきストアされるハッシュテーブルに連結リストという形式でコンテンツがストアされる.
リスト構造で延長するコンテンツの識別には,ドメインIDを識別子として利用する.
ドメインIDは,コンテンツIDと同様のハッシュアルゴリズムを用いて算出されるメッセージダイジェストの先頭32bitで表現される,ドメイン名を引数として生成される識別子である.
例えば,ドメイン名を``www.example.com"とする場合,そのメッセージダイジェストが``86ff20100c058b857bae9785bf0267e6c6afb740c18b8e9a87258485"であるとすると,``86ff20100c058b857bae9785bf02"がドメインIDとなる具合である.
このように算出されたドメインIDは,DNSのQuestion Sectionのうち,それぞれ16bit分の領域を持つタイプとクラスの領域に埋め込まれる.
上記の仕組みによって,コリジョンが発生した際には,ドメインIDをキーとしてコンテンツを識別する.

メッセージダイジェストの出力長はそれぞれ,コンテンツIDが56byte,ドメインIDが28byteである.
ここで,ドメイン名が``www.example.com"でAのレコードタイプの組み合わせを考える.
コンテンツIDは,``www.example.comA"のように文字列和をメッセージとする.
ドメインIDは,ドメイン名``www.example.com"がメッセージとなる具合である.

\subsubsection{ゾーン分割}
本項では,ゾーンの分割方法およびマネージャノードのアドレスとそのゾーンの範囲に関する対応表について説明する.
はじめに比較のために,従来のシステムの場合について説明する.
従来のシステムでは,ドメインの階層構造に従い,ドメインの管理ノードを下位のドメイン管理ノードに委譲することでゾーンが分割される.
この仕組みでは,ゾーン内の全てのレコード情報はゾーンファイルに画一的にまとめられ,そのゾーンを管理する権威サーバがレコード情報の保持機能とクライアントから応答するという二つの機能を担う.
このゾーン分割メソッドでは,レコード情報の帰属が明確であり,ドメインの管理ノードがトラストアンカーとしての役割を同時に担うことができるメリットがある.
一方のDNS-TDでは,識別子を算出する際に使用するハッシュ関数によって構成される名前空間に基づき,ソートされたハッシュの名前空間の連続した範囲で分割する.
この分割された連続した範囲に基づきゾーンがマネージャに割り当てられることで,既存システム同様にレコード情報全体を分散的に管理する.

上記で説明するように,マネージャが管理するゾーンは,ハッシュの名前空間の連続した一部の範囲である.
従って,レコード情報は,ハッシュの名前空間上で識別子をソートした際に,帰属する範囲を管理するマネージャによって保持される.
マネージャのアドレスを解決する方法には,ゾーンとしてハッシュ値の範囲とそのマネージャおよびマネージャのアドレスに関する対応表~\ref{tab:hash-management}によって解決される.
DNS-TDでは,全てのサービスノードがこの対応表を保持できることを想定しており,ノードは識別子に基づきどのマネージャがコンテンツを保持しているのかを一意に特定する.
\begin{table}[htb]
 \caption[マネージャとゾーンの対応表]{6つのマネージャによって管理されるハッシュテーブルにおいて,マネージャの情報とそのマネージャが管理するゾーンが記載された対応表の例}
 \centering
  \begin{tabular}{rrr}
    \toprule
    \begin{tabular}{c}\textbf{ゾーン}\end{tabular} & \begin{tabular}{c}\textbf{マネージャアドレス}\end{tabular} & \begin{tabular}{c}\textbf{ドメイン}\end{tabular} \\
    \midrule
   % ... & ... & ... \\
   % (30…00, 4z…zz) & 192.35.51.30 & com.  \\
   % ... & ... & ... \\
   % (90…00, az…zz) & 192.5.6.30 & net. \\
   % ... & ... & ... \\
   % (f0…00, gz…zz) & 199.249.112.1 & org. \\
   % ... & ... & ... \\
   % (l0…00, mz…zz) & 213.248.216.1 & uk. \\
   % ... & ... & ... \\
   % (r0…00, sz…zz) & 199.254.31.1 & info. \\
   % ... & ... & ... \\
   % (w0…00, xz…zz) & 194.0.0.53 & de. \\
   % ... & ... & ... \\
    (00…00, 4z…zz) & 192.35.51.30 & com.  \\
    (50…00, az…zz) & 192.5.6.30 & net. \\
    (b0…00, gz…zz) & 199.249.112.1 & org. \\
    (h0…00, mz…zz) & 213.248.216.1 & uk. \\
    (n0…00, sz…zz) & 199.254.31.1 & info. \\
    (t0…00, zz…zz) & 194.0.0.53 & de. \\
    \bottomrule
  \end{tabular}
 \label{tab:hash-management}
\end{table}


\newpage
\subsection{リソースレコード}
\subsubsection{認証基盤}
\label{sec:certificate}
% 認証局を導入するとインターネットの匿名性を実現することが難しくなるのではないか
% レコード情報に認証局を導入する場合,webなどのサービスやコンテンツを提供するのが少し困難になるではないかという懸念
% githubなどにおいては,同一ドメインにファイルという形でユーザにディスクを提供している
% CAを委譲する仕組みがある.これによって,
本項では,レコード情報の信頼性担保のための認証基盤について説明する.
\begin{figure}[h]
 \centering
 \includegraphics[scale=0.7]{figure/certificate-procedure.png}
 \caption{レコード情報操作におけるプロセスの概略図}
 \label{fig:manager-provider}
\end{figure}

DNS-TDでは,全てのコンテンツについて,信頼される第三者からストアしても良いと認可されていることを前提としている.
すなわち,ハッシュテーブル上のコンテンツへの操作,またはコンテンツをハッシュテーブル上にストアするなどの操作処理をする際,プロバイダは,信頼される第三者からのレコード情報に操作することを認可してもらう必要がある.
認可の証明書を発行する認証局は,プロバイダの基本情報とレコード情報に基づき証明書の発行を決定する.
いま,ドメイン名``www.example.com"のリソースレコードタイプAとして``93.184.216.34"というレコード情報を関連づけるとする.
プロバイダは,認証局に対して証明書発行リクエストを転送する.
認証局は,リクエストされたレコード情報についてIPアドレスの到達性と不審な文字列が含まれていないこと,利用目的について評価を施す.
認可された場合には,その証にディジタル証明書を発行し,マネージャにストアリクエストを転送する.
マネージャは,ディジタル証明書に付与された署名に基づきコンテンツの完全性を評価し,認証された場合コンテンツIDを計算し,担当マネージャにストアリクエストを転送する.
上記の手続きを経たコンテンツがハッシュテーブル上にストアされる.


\subsubsection{レコードタイプ}
本項では,DNS-TDで使用するリソースレコードのタイプについて説明する.

% 提案システムで使用するレコードタイプを概観するか,もしくは既存システムにおけるリソースレコードのタイプの課題から説明するのがいいだろう
はじめに,DNS-TDにおけるDNSSECの位置づけについて述べる.
DNSSEC~\cite{rfc4033}は,権威サーバからの応答パケットの偽装を検知することを目的として,データの作成元の確認とデータの完全性および,不在情報応答情報の証明するDNSの拡張仕様である.
これは,主としてDNSの応答パケットを偽装できる程度のパラメータであることに起因する.
他方で,DNS-TDでは,応答パケットに224bitのメッセージダイジェストが含まれるため,悪意のある応答パケットをフルサービスリゾルバに意図的にキャッシュすることは極めて困難である.以上の理由から,DNS-TDではDNSSECの目的にそぐわないため,リソースレコードとして使用されない.

次に,DNSSEC以外のリソースレコードについて説明する.
第~\ref{sec:dns-infiltration}項で示すように,既存の名前解決システムでドメインに関連づけることができるリソースレコードのいくつかのタイプは,DNS Infiltrationとして機能することができる.
DNS Infiltrationを抑止するリソースレコードであることの必要条件は,ドメインに関連のない任意の文字列がレコード情報に含められないことである.
既存のDNSのリソースレコードのタイプのうち,任意の文字列を含めることができるのタイプは以下の通りである.


表~\href{tab:infil-rtype}のDNS Infiltrationとして機能する可能性のあるリソースレコードのタイプのうち,IPアドレスを偽装して情報を転送するものについては,第~\ref{sec:certificate}項で述べた認証基盤によってレコード情報の正当性評価でスクリーニングすることができる.
他方で,任意の文字列を注入できるのが,NULL・TXT・CNAMEタイプである.

NULLについて考える.
NULLタイプの目的は,実験用と定義されている~\cite{rfc1035}.
TXTについて考える.
CNAMEについて考える.
ホスト名に対する別名で関連づけることができるCNAMEは,一つのサーバにおいてサービスごとにサーバの名前を変更させるために使用される.
%DNS-TDでは,ドメインごとにゾーンは保持しないので

% データベースについて
% Redisについて説明

%\subsection{データベース}
\subsubsection{コンテンツのデータフォーマット}
本項では,マネージャにて管理されるコンテンツのフォーマットについて説明する.

\begin{figure}[h]
 \centering
 \includegraphics[scale=0.6]{figure/content-file.png}
 \caption{コンテンツのデータフォーマット}
 \label{fig:manager-provider}
\end{figure}

%\subsection{動作メカニズム}
%\subsubsection{レコード情報に対する操作}
%本項では,DNS-TDにおいて使用されるリソースレコードのタイプと
%%TTLの更新方法について説明する
%
%\subsubsection{名前解決}
%ハッシュテーブルのレプリケーション手法
%特定のハッシュ範囲を管理するノードは,複数用意させ,そのアドレスを対応表に明記し,ストアする際にその全てのレプリケーションサーバにストアリクエストする

\section{評価}
\label{sec:evaluation}
%評価基準
%Fialure Resilience, Performance(Latency, Misconfiguration, Load Balance)
%サービスとして要求されるもの
% Resolution response time
% Resolution Accuracy
% Resolution guarantee
% Resolution fairness
%
%システムとして要求されるもの
% スケーラビリティ
% マネージビリティ
% Deployed System
% Fault Torrelance
%
%セキュリティ
% アクセシビリティ
% Aurhontication
% Data Confidentialy
% プライバシー保護
% ロバストネス
% ネットワークプライバシー
本章では,提案システムのプロトタイプ上でのトンネリング実装の動作に基づき,権威サーバにおける機能の分離と名前空間をフラットにするメソッドがトンネリング抑止に機能することを明らかにする.
また,提案システムの特性に関して,既存のDNSとの比較に基づき評価し,その特性を
トラフィック量の緩和と名前解決を高速化する有意性を示すことを説明する.

システムのプロトタイプには,Python3を用いて,フルサービスリゾルバとマネージャ,プロバイダを実装した.
DNSトンネリングに対する抑止効果については,第~\ref{sec:eval-tunnel}節で述べ,システムの特性については,第~\ref{sec:eval-feature}節でそれぞれ詳細に述べる.


%はじめに,提案システムのフルサービスリゾルバが満たす機能の要素は,以下の通りである.
%\begin{itemize}
% \item スタブリゾルバからの名前解決クエリに対して,コンテンツIDとドメイン名IDを算出する機能
% \item コンテンツIDに基づき,マネージャを探索する機能
% \item スタブリゾルバからの名前解決クエリをSORESのフォーマットに変換させ,マネージャに転送する機能
% \item マネージャからの応答パケットをDNSのフォーマットに変換させ,送信元のスタブリゾルバに転送する機能
%\end{itemize}
%
%以上の機能を提供するサービスのアルゴリズムを以下のように設計した.
%
%%client.addr = address
%client.port = port
%client.data = payload
%clinet.data.header
%client.data.qname
%client.data.rtype
%hash.sha3_224()
%hash.md5()
%map.range.start : list
%map.range.end : list
%map.addr : list
%map.name : list
%zone_x


\begin{algorithm}[!h]
 \caption{スタブリゾルバからのクエリ処理}
 \label{algo:full-service}
  \SetKwProg{Fn}{}{\string:}{}
  \SetKwFunction{Calc}{calculate\_id}
  \SetKwFunction{Find}{find\_manager}
  \SetKwFunction{Query}{query\_content}
  \SetKwFunction{Transfer}{transfer\_answer}
 $qname \leftarrow client.data.qname$\;
 $rtype \leftarrow client.data.rtype$\;
 $start \leftarrow map.range.start$\;
 $end \leftarrow map.range.end$\;
 $\vspace{-0.3cm}$\;

 Calculate the Content ID and the Domain ID\;
 \Fn{\Calc{qname, rtype}}{
   $content\_id \leftarrow hash.sha3\_224(qname+rtype)$\;
   $domain\_id \leftarrow hash.md5(qname) / 2$\;
   $return \ content\_id,\ domain\_id$
 }


 $\vspace{-0.3cm}$\;
 Find a manager who has zone includes the Content ID\;
 \Fn{\Find{start, end}}{
   \For {$i,\ j\ \textbf{in}\ map.range.start,\ map.range.end$} {
     \If {$i \leq content\_id \leq j$} {
       $p \leftarrow map.range.start.index(i)$\;
       $manager\_addr \leftarrow map.addr[p]$\;
       $break$\;
     }
   }
 }
 $\vspace{-0.3cm}$\;

 Query the content to the manager\;
 $answer \leftarrow query(manager\_addr,\ content\_id,\ domain\_id)$\;
 $\vspace{-0.3cm}$\;

 Transfer the answer to Client\;
 $transfer((client.addr,\ client.port),$\;
 $\qquad\qquad client.data.qname,\ answer.header.rcode,\ answer.rdata)$\;
\end{algorithm}

%%このアルゴリズムの説明は,本来提案手法の章で動作説明中に述べるべきであろう.
%
%
%
%% フルサービスリゾルバとプロバイダをいかにして判別するのか
%
%\begin{description}
% \item[マネージャ]\mbox{}
%  \begin{itemize}
%   \item プロバイダからのコンテンツストアリクエストに対して,コンテンツIDを探索し担当マネージャノードにコンテンツ保持をリクエストする機能
%   \item コンテンツを管理するためのデータベースを保持する機能
%   \item フルリゾルバからの名前解決クエリに対して,そのコンテンツIDに一致するコンテンツを探索し応答する機能
%  \end{itemize}
% \item [プロバイダ]\mbox{}
%  \begin{itemize}
%   \item コンテンツに対して操作する機能
%  \end{itemize}
%\end{description}

\subsection{DNSトンネリング}
\label{sec:eval-tunnel}
\subsubsection{検証環境}
DNSトンネリング抑止機能を評価するために用いるトンネリング実装は,〜を用いる,
検証環境には,Dockerを使用し,コンテナ同士の接続によってDNSのチェーンを実現する.
その概略図は,以下の通りである.

%提案システム上で,iodineを動作させた際のパケットキャプチャーした様子を説明する
\subsubsection{DNS Exfiltration}
本項では,提案システムのDNS Exfiltrationへの機能性について評価した結果を述べる.
評価には,実装したシステムを
\subsubsection{DNS Infiltration}


\subsection{要素ごとの特性}
\label{sec:eval-feature}
本節では,DNSとの比較評価に基づいた提案手法の特性について説明する.
提案システムと既存システムの名前解決における差分には,表~\ref{tab:diff_sores_dns}のようなものが予想される.
\begin{table}[htb]
 \centering
  \begin{tabular}{ccc}
    \toprule
     & \textbf{DNS} & \textbf{SORES} \\
    \midrule
    \textbf{ドメイン長} & \begin{tabular}{c}変長\\(最大253byte)\end{tabular} & \begin{tabular}{c}固定長(72byte)\\(コンテンツID(56) \& \\ ドメインID(16))\end{tabular} \\ \hline
    \textbf{\begin{tabular}{c}トランザクション\\試行回数\end{tabular}} & ラベル数回 & 1 \\ \hline
    \textbf{RTT} & \begin{tabular}{c}全ての権威サーバ\\とのRTT総和\end{tabular} & マネージャとのRTTのみ  \\ \hline
		 \textbf{ゾーンファイル} & ファイル & インメモリデータベース\\ \hline
		 \textbf{その他} & & \begin{tabular}{l}・ハッシュ計算の発生\\・マネージャ探索\end{tabular}\\
    \bottomrule
  \end{tabular}
 \caption{DNSとSORESの特性比較}
 \label{tab:diff_sores_dns}
\end{table}



\begin{description}
 \setlength{\itemsep}{0pt}
 \setlength{\leftskip}{1.0cm}
 \item[RTT:] 名前解決の速度
 \item[トラフィック量:] ネットワーク全体への影響度
\end{description}

上記の特性に関する評価にあたり,従来のDNSトラフィックのデータセットと提案システムにおける名前解決時のトラフィックデータセットを用意した.
これらデータセットについて,統計アプローチに基づいて,評価を行った結果を示す.


SORESでは,スタブリゾルバからの名前解決クエリのうち,そのヘッダー情報に基づき識別子を算出することで,この識別子に関連づけられたレコード情報を操作する仕組みに基づき名前解決機能が動作する.
DNSでは,ゾーンをドメインごとに分割し,フルサービスリゾルバはルート権威サーバから目的の権威サーバに向かって再帰的に問い合わせ,権威サーバのアドレスを解決しながら,最終的にコンテンツを保持する権威サーバからコンテンツを取得する.
既存のDNSとSORESの特性を表~\ref{tab:diff_sores_dns}に示す.


から識別子に基づき,コンテンツを保持するサーバに名前解決リクエストを投げる.
SORESは,フルサービスリゾルバにてスタブリゾルバから問い合わせられた名前解決命令に基づきレコード情報の識別子を算出する.
この識別子に基づき,フルサービスリゾルバからは,1ホップでコンテンツを保持するサーバに問い合わせる仕組みで動作する.
\subsubsection{RTT(Round Trip Time)}
本項では,SORESが識別子から一意にレコード情報にアクセス可能である点について,DNSとの比較評価を行う.
SORESの名前解決では,
従来のDNSでは,ラベルごとにゾーンが移譲されている場合,レコード情報を保持するノードまでのホップ数はラベル数nに比例する.
それに対して,SORESでは,識別子から一意にレコード情報を保持するマネージャノードを特定できるので,常にホップ数は2である.
このため,既存の名前解決システムより速度の向上が期待される.
% 実装の必要はない
% 既存のDNSにおけるラウンドトリップのうち,再帰問い合わせの最後の権威サーバRTTがSORESのフルリゾルバとマネージャのそれになる.
% 遠いものと近いもののRTTを用意する必要がある
\subsubsection{パケットサイズ}
% 実装する必要はない.理論で評価できる
% あえてするとすれば,フルサービスリゾルバとマネージャ間通信
% DDoSへの影響については,リクエストとレスポンスパケットのサイズからアンプ率に着目する
% アンプ率は,既存のDNSと基本的に変わらない.DNSSECがあるかないかについて議論するくらい
SORESでは,56byteを固定長とするコンテンツIDをシンボルとすることによって,レコード情報にアクセスする.
この仕組みの影響で,SORESのパケットは従来のパケットと比較して肥大する特性がある.
このため,送信元を目的ホストと偽装することで目的ホストの計算リソースを圧迫するDDoS攻撃に対して,脅威を高める可能性が予想される.
\subsubsection{トラフィック量}
% SORESについては,実装の必要性はない
% ただし,DNSについてトラフィックデータを収集する必要がある
% 必要なデータ
% * 一定期間のSORESトラフィック
% * 同一期間のDNSトラフィック
% 比較項目
% * 平均したドメイン長と
本項では,提案手法におけるトラフィックについて評価する.
評価において,
SORESでは,シンボル志向の名前解決メソッドによって,既存の再帰問い合わせによるメソッドよりも少ないトラフィックに抑えることが期待される.
SORESでは,クエリ数とトラフィック量は比例関係にある.
他方で,新たにマネージャ間通信という従来にはないトラフィックが発生する.
本項では,これトラフィックがネットワーク全体にどの程度影響を及ぼすのかについて評価する.
% マネージャが攻撃された場合の影響は,世界規模に影響する.
% 従来のドメインごとにゾーンが分離している設計と違い,一つのゾーンには様々な組織のドメインが管理されている.
% このゾーンを全て書き換えることができることは脅威である.


\section{考察}
本章では,第\ref{sec:evaluation}で行なった評価を踏まえて,提案手法の有用性について考察する.
% 機能を分離させることによって,クエリを用いてデータを転送する仕組みを無効化することを実現
% 他の名前解決システムと違い,クライアントサイドへの変更点が少ないためマイグレーションが期待される
% 
\subsection{提案手法におけるバイパス手法}
\subsection{今後の課題}
% マネージャノードが攻撃されると,被害範囲が大きい
% プロバイダからのコンテンツ操作リクエストパケットに不正なクエリをトリガーとして,マネージャノードが攻撃されるリスク
% - プロバイダからの問い合わせをパースする際にセキュリティ機構を備えることによって,対策を講じる必要性がある
% Redisサーバをオープンにすることはない

%\subsection{マネージャノードの最適な数}
%\subsection{検閲}
%提案システムでは,ドメイン名とそれに関連付けれたレコード情報のペアに基づき構成されるコンテンツは,フラットな名前空間に一意にマップされる.
%コンテンツは,ソートされたハッシュ値の特定の範囲ごとに分割され,その範囲をゾーンとしてマネージャサービスによって管理される.

%\subsection{コンテンツIDを計算する最適ノード}

\section{結論}
% 機能を分離させることによって,クエリを用いてデータを転送する仕組みを無効化することを実現
% 他の名前解決システムと違い,クライアントサイドへの変更点が少ないためマイグレーションが期待される
本論文では,DNSトンネリングを抑止する名前解決システムを提案した.
DNSトンネリングが動作する仕組みを分析し,修正する対象として,スタブリゾルバから権威サーバまでクエリパケットが転送される仕組みに着目した.
提案システムDNS-TDは,権威サーバのコンテンツを保持およびクライアントからの問い合わせに応答する機能とコンテンツを編集する機能を2つサービスノードに分割させる設計をとった.
これによって,スタブリゾルバと任意の権威サーバ間でデータがやりとりされることが抑制され,DNSトンネリング通信発生の抑止に寄与する.
%また,DNSは現在のインターネットを支える基盤システムとして根付いており,大幅な修正が加えられた場合,接続障害が発生する可能性がある.
%提案システムは,フルサービスリゾルバと権威サーバにおける名前解決の仕組みのみの変更に抑えているため.現実的なマイグレーションが期待される.
評価では,実装した提案システムのプロトタイプ上での擬似トンネリング通信を発生させるシミュレーションテストに基づいて,トンネリング通信の抑止に対して本提案システムが有用であることを示した.
また,特性評価の結果から,提案システムの名前解決におけるクエリ転送回数がスタブリゾルバからのサーバまで2回である点は,他のどの名前解決システムよりも少ない.
すなわち,提案システムが高速な名前解決とトラフィック量を削減できるという特性を持っていることを明らかにした.
%また,提案システムの特性の評価を行い,提案システムが既存システムよりも少ないトラフィックでかつ高速な名前解決を実現できるという優位性があることを示した.

提案システムは,ドメインに依らないフラットな名前空間の範囲に基づきゾーンを分割する設計上,1つのサーバが保持するゾーンには複数のドメインが含まれる.
そのため,既存システムにおける権威サーバの障害では,その権威サーバが管理するドメインのみに影響が及ぶのに対し,提案システムではインターネット全体に影響が及ぶ潜在的な脅威がある.
今後の課題は,この脅威の対処として負荷分散の仕組みを検討することである.




% ===================謝辞===================
\newpage
\acknowledgements
%ご指導ご鞭撻賜りありがとうございました.
本研究をまとめるにあたり,審査を引き受けて下さいました本学情報科学領域の門林雄基教授,笠原正治教授,林優一教授,妙中雄三准教授に深く感謝申し上げます.

%先生方へ
% 機会
主指導教員の門林雄基教授におきましては,セキュリティ領域における高い視点から研究の方向性を示して下さいました.
また,先端科学をテーマにする研究者としての姿勢や社会人としての身の振り方,修練し続けることの意義を親身になって御助言して下さいました.
笠原正治教授におきましては,考察できていなかった課題を的確に御指摘して下さいました.
林優一教授におきましては,本学で学ぶより前から親身に気遣って下さり,本研究に対してセキュリティの専門の立場から鋭くかつ建設的な御指摘を下さいました.
%わかりやすさ,行間を読ませないプレゼン,スケジュール管理
妙中雄三准教授におきましては,Weekly Meetingという形で貴重な御時間を多く頂戴し,熱心に討論に付き合って下さいました.
常に妥協のない姿勢で鋭く御指導して下さり,説明に関する方法論など研究のいろはを丁寧に御教示して下さいました.
また,樫原茂助教授,Doudou Fall助教授には,日頃より相談に乗って下さり,研究に取り組む姿勢や考え方について御助言して下さいました.
%学生
研究室の学生メンバーにおきまして,先輩は様々な形で研究を導いて下さり,唯一の同期はそのキャラクターでいつも楽しませてくれ,後輩は執筆時の添削やプレゼン練習に時間を縫って付き合うなど支援して下さいました.
%お陰様で,明るく健康的な研究活動を行うことができました.
研究室の秘書様方におきましては,研究会への手続きや事務手続きなどの研究活動を支えて下さいました.

%何を学んできたか
サイバーレジリエンス構成学研究室は,日常的に英語を話す機会があるほどに国際的で,建設的な討論を通じて先端研究を行える環境にありました.
%家族へ
両親含め家族からは,奈良という慣れない土地で学ぶことを気遣いながら,どんな時も温かく優しく見守って下さいました.
その眼差しに恥じぬようにと努めることで,本研究をまとめることが出来ました.
このような環境と周囲との関わりなくして本研究はまとまりませんでした.
改めまして,本研究に関わって下さいました皆様に重ねて御礼申し上げます.




% ===================参考文献===================
\newpage
\bibliographystyle{ieeetr}
\bibliography{mthesis}


% ===================付録===================
\appendix

\section{レコードタイプの使用分布}
\begin{table}[htb]
 \caption{レコードタイプの分布}
 \centering
  \begin{tabular}{lrr}
    \toprule
    \multicolumn{1}{c}{\textbf{Type}} & \multicolumn{1}{c}{\textbf{パケット数}} & \multicolumn{1}{c}{\textbf{割合}} \\
    \midrule
    A & 236,210,050 & 54.778 \\
    AAAA & 149,322,427 & 34.629 \\
		PTR & 43,060,608 & 9.986 \\
		SRV & 1,497,622 & 0.347 \\
		MX & 474,827 & 0.110 \\
		ANY & 281,023 & 0.0657 \\
		SOA & 226,975 & 0.053 \\
		TXT & 115,300 & 0.027 \\
		NS & 12,028 & 0.003 \\
		TKEY & 4518 & 0.001 \\
		NAPTR & 4281 & 0.001 \\
		SPF & 512 & 0.000 \\
		CNAME & 196 & 0.000 \\
		AXFR & 2 & 0.000 \\
		NULL & 2 & 0.000 \\ 
		合計 & 431,210,371 & 100.000 \\
    \bottomrule
  \end{tabular}
 \label{tab:distribution-rtype}
\end{table}


\newpage
\section{発表リスト(口頭発表)}
%\subsection{国内研究会}
\begin{enumerate}
 \item \underline{高須賀 昌烈}, 妙中 雄三, 門林 雄基, ``非実在ドメインに対するネガティブキャッシュの拡張と再帰問い合わせハッシュ化の提案," 電子情報通信学会 情報ネットワーク研究会, 2019-10-ICTSSL-IN, 2019年10月.
\end{enumerate}

\end{document}
