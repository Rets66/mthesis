\section{考察}
本章では,第\ref{sec:evaluation}で行なった評価を踏まえて,提案手法の有用性について考察する.
% 機能を分離させることによって,クエリを用いてデータを転送する仕組みを無効化することを実現
% 他の名前解決システムと違い,クライアントサイドへの変更点が少ないためマイグレーションが期待される
% 
\subsection{提案手法におけるバイパス手法}
% マネージャノードが攻撃されると,被害範囲が大きい
% プロバイダからのコンテンツ操作リクエストパケットに不正なクエリをトリガーとして,マネージャノードが攻撃されるリスク
% - プロバイダからの問い合わせをパースする際にセキュリティ機構を備えることによって,対策を講じる必要性がある
% Redisサーバをオープンにすることはない

%\subsection{マネージャノードの最適な数}
%\subsection{検閲}
%提案システムでは,ドメイン名とそれに関連付けれたレコード情報のペアに基づき構成されるコンテンツは,フラットな名前空間に一意にマップされる.
%コンテンツは,ソートされたハッシュ値の特定の範囲ごとに分割され,その範囲をゾーンとしてマネージャサービスによって管理される.

%\subsection{コンテンツIDを計算する最適ノード}
