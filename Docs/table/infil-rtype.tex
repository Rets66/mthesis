\begin{table}[h]
 \caption[DNS Infiltrationに使用されうるレコードタイプ]{DNS Infiltrationとして使用することができるレコードタイプの一覧}
 \centering
  \begin{tabular}{lrll}
    \toprule
		\multicolumn{1}{c}{\textbf{タイプ}} & \multicolumn{1}{c}{\textbf{\begin{tabular}{c}最大サイズ\\(byte)\end{tabular}}} & \multicolumn{1}{c}{\textbf{説明}} &\multicolumn{1}{c}{\textbf{実装}}\\
    \midrule
		A & 4 & \ ホストのIPv4アドレス &\\ \hline
		NS & 4 & \ 権威サーバ & \, \cite{dnscat2}\\ \hline
    %MF & 4 & メール転送サーバ \\
		CNAME & 253 & \ 別名 & \, \cite{iodine},\cite{dnscat2}, \cite{dnscapy}, \cite{tuns}\\ \hline
		%SOA & 253 & 権威ゾーンの開始 & \\
		NULL* & 255 & \ NULL(実験用) & \, \cite{iodine}\\ \hline
		PTR & 4 & \begin{tabular}{l}ドメイン名のポインター\\(逆引き)\end{tabular} & \\ \hline
    %HINFO & 13 & ホスト情報 \\
    %MINFO & 14 & メールボックスおよびメールリスト情報 \\
		MX & 253 & \ メール交換 & \, \cite{iodine},\cite{dnscat2}\\ \hline
		TXT* & 255 & \ 任意文字列 & \begin{tabular}{l}\cite{iodine},\cite{dnscat2}, \cite{denise}, \cite{dns-shell},\\ \cite{dnscapy}, \cite{dohtunnel}, \cite{dohc2}\end{tabular}\\ \hline
		AAAA & 32 & \ ホストのIPv6アドレス & \\ \hline
		SRV & 180 & \begin{tabular}{l}ドメイン名に対する\\サービスの場所\end{tabular} & \, \cite{iodine}\\ \hline
		 DNSKEY & 40 & \ DNSSECのための公開鍵 & \, \cite{dns2tcp}\\
    %TLSA & 52 & \\
    \bottomrule
  \end{tabular}
 \label{tab:infil-rtype}
\end{table}
