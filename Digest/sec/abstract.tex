\section{概要}
 巧妙化するサイバー攻撃の手法の中に,攻撃通信を無害な通信に偽装することで検知を迂回する手法がある.
 DNSトンネリングと呼ばれる手法は,そのような秘匿通信手法の中で最も広く利用されている.
 この手法に対して,従来は,サブドメインの長さおよびエントロピー,トラフィック頻度などを特徴量とする閾値や悪性モデルを推定する検知アプローチが取られてきた.
 しかし,転送データ量の削減やパケット間のインターバルを長期化させるといった転送効率を下げる手法を用いることで,既存の検知アプローチを迂回される脅威がある.

 この脅威に対して,本研究ではDNSトンネリングの発生を抑止する名前解決システムを提案する.
 システムに採用したアプローチは,権威サーバの機能を分離させることでスタブリゾルバからサーバへのクエリ透過を抑制し,このメカニズムによってDNSトンネリングの発生を抑止することができる.
 評価では,実装した提案システムのプロトタイプ上でトンネリング通信をシミュレーションさせることによって,DNSトンネリングの通信抑止に有効であることを示した.
 また,既存システムとの比較に基づいた特性の評価を行った結果から,提案システムにはトラフィック量の削減と高速な名前解決という優位性があることを示した.
