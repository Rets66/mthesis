\section{結論}
% 機能を分離させることによって,クエリを用いてデータを転送する仕組みを無効化することを実現
% 他の名前解決システムと違い,クライアントサイドへの変更点が少ないためマイグレーションが期待される
本論文では,DNSトンネリングを抑止する名前解決システムを提案した.
DNSトンネリングが動作する仕組みについて分析し,スタブリゾルバからコンテンツを操作する権威サーバまで問い合わせパケットが転送される点に着目した.
%DNSトンネリングが動作する仕組みについて分析し,再帰問い合わせの仕組みでは,スタブリゾルバからのクエリは権威サーバにトが転送される点に着目した.
提案システムDNS-TDでは,権威サーバの機能をコンテンツを保持しクライアントからの問い合わせに応答する機能とコンテンツを編集する機能を2つサービスノードに分割させる.
この設計によって,スタブリゾルバからの問い合わせパケットを受信するノードを作成することが防ぐことで,トンネリング通信の発生を抑止する.
また,提案システムの機能特性について評価を行い,既存システムよりも少ないトラフィックでなおかつ高速な名前解決を実現できるという優位性を持っていることを示した.
DNSは現在のインターネットの根幹技術であり,基盤システムに根付いているため,既存システムから完全に移行することは困難である.
この点を踏まえて,提案システムでは,フルサービスリゾルバと権威サーバにおけるトランザクションのみの変更に抑えることで.既存システムからの移行を実現しやすいものに設計した.

今後の課題として,最適なマネージャノードの数を推定することがある.
これには,マネージャが保持するコンテンツの数とその計算負荷の関わりについて評価する必要がある.
また,現在のDNSのルート権威サーバに適用されるようなAnycast技術を用いることで,地理的制約を解消し少ないマネージャ数で運用できる可能性がある.
この課題について検討を行っていく.
