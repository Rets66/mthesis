\section{考察}
本章では,第\ref{sec:evaluation}で行なった評価を踏まえて,提案手法の課題と有用性について考察する.

\subsection{提案手法におけるバイパス手法}
%\subsection{マネージャノードの最適な数}
%\subsection{ハッシュ関数の寿命}
%\subsection{実環境導入への課題}
% マネージャノードが攻撃されると,被害範囲が大きい
% マネージャが攻撃された場合の影響は,世界規模に影響する.
% このゾーンを全て書き換えることができることは脅威である.
% プロバイダからのコンテンツ操作リクエストパケットに不正なクエリをトリガーとして,マネージャノードが攻撃されるリスク
% 提案システムでは,全てのネットワークが統一的な名前解決基盤を使用することになるため,国家レベルで検閲が実施されている環境には極めて大きな影響を及ぼす
\subsection{今後の課題}
% マネージャノードが悪意のあるユーザによって攻撃された際については,議論する必要があるのか
% - プロバイダからの問い合わせをパースする際にセキュリティ機構を備えることによって,対策を講じる必要性がある

